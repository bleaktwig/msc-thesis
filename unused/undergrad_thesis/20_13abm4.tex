\subsection{Fourth Order Adams-Bashforth-Moulton Method} \label{add:abm4}
The Fourth Order Adams-Bashforth-Moulton method or simply Adams-Bashforth-Moulton 4 (ABM4) is a multi-step method used to solve initial value problems similar to RK4 in use.
Solving the same equation as RK4, let an initial value problem for estimating $\mathbf{x}(z+h)$ be:

    \begin{equation*}
        \mathbf{\dot{x}} = \mathbf{f}(z,\mathbf{x}), \mathbf{x}(z_0) = \mathbf{x}_0\,,
    \end{equation*}

where the definitions are equivalent to the ones given in Section \ref{ssec:framework_rk4}.
To solve the problem using a step size $h$, first the fourth order Adams-Bashforth (AB4) method is proposed:
    \begin{equation}
        \label{eq:abm4_explicit} \mathbf{x}_{n+1}^* = \mathbf{x}_n + \frac{h}{24} (55\cdot \mathbf{f}(z, \mathbf{x}_n) - 59\cdot \mathbf{f}(z-h, \mathbf{x}_{n-1}) + 37\cdot \mathbf{f}(z-2h, \mathbf{x}_{n-2}) - 9\cdot \mathbf{f}(z-3h, \mathbf{x}_{n-3}))\,,
    \end{equation}
which is commonly denoted as the ``explicit'' AB4 method.
Then, from this result it's possible to compute a more accurate result for $\mathbf{x}_{n+1}$ using the fourth order Adams-Moulton (AM4) method, or the ``implicit'' AM4 method:
    \begin{equation}
        \label{eq:abm4_implicit} \mathbf{x}_{n+1} = \mathbf{x}_n + \frac{h}{24} (9\cdot \mathbf{f}(z+h, \mathbf{x}_{n+1}^*) + 19\mathbf{f}(z,\mathbf{x}_n) - 5\mathbf{f}(z-h,\mathbf{x}_{n-1}) + \mathbf{f}(z-2h,\mathbf{x}_{n-2})\,.
    \end{equation}
While the results of this method are not necessarily more accurate than the ones obtained by RK4 since both methods have the same convergence order, % Note: This could use a reference or an addendum explaining convergence order.
the method has the advantage that it only requires to compute $\mathbf{f}(\cdot,\cdot)$ twice for every time-step (assuming that the previous results are stored), whereas RK4 requires this function to be computed four times per time-step.
This can result in a serious improvement if the time it takes to compute $\mathbf{f}(\cdot,\cdot)$ is large.

One problem that arises from the implementation though is that at the beginning of the computation only $\mathbf{x}_0$ is known, and four steps are needed to start.
To solve this, it's not uncommon to use RK4 to compute $\mathbf{x}_1$, $\mathbf{x}_2$ and $\mathbf{x}_3$ and then start using ABM4.