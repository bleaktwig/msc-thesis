\subsection{Sherman-Morrison Formula} \label{add:shermanmorrison_formula}
Originally proposed by Jack Sherman and Winifred J. Morrison~\cite{sherman1950adjustment}, the Sherman-Morrison formula says that if $A\in\mathbb{R}^{n\times n}$ is a nonsingular matrix and $\mathbf{u}, \mathbf{v} \in\mathbb{R}^n$ are vectors:
    \begin{equation*}
        (A + \mathbf{u}\mathbf{v}^T) = A^{-1} - \frac{A^{-1}\mathbf{u}\mathbf{v}^T A^{-1}}{1 + \mathbf{v}^T A^{-1} \mathbf{u}}\,,
    \end{equation*}
given that $1 + \mathbf{v}^T A^{-1} \mathbf{u} \neq 0$.

\newpage

This formula is especially useful when $A^{-1}$ is unknown but $A$ is known, or in special cases such as the one seen in Equation \eqref{eq:mhc_sm_det}, where:
    \begin{align*}
        P_{k|k} &= (P_{k|k-1}^{-1} + \mathbf{h}\mathbf{h}^T)^{-1}\\
        &= P_{k|k-1} - \frac{P_{k|k-1} \mathbf{h} \mathbf{h}^T P_{k|k-1}} {1 + \mathbf{h}^T P_{k|k-1} \mathbf{h}}\,,
    \end{align*}
since it eliminates the need to compute the inverse of $P_{k|k-1}$ and, if the operations are resolved in a particular order, it even eliminates the need for multiplying matrices, which is an expensive operation by itself. 
% Note: If I add an addendum "On the complexity of operating matrices" this should ref there.