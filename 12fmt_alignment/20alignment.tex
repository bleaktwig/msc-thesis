% !TEX root = ../main.tex
\subsection{FMT Alignment}
\label{12.20::fmt_alignment}
% --+ Why is it needed +--------------------------------------------------------
    In an ideal scenario, the target and each detector would be installed precisely in their required positions.
    However, in the real world, there are inevitable misalignments in their placements.
    These misalignments must be accounted for and incorporated into the reconstruction process to ensure meaningful results.

    Within the CLAS collaboration, the Calibration and Commissioning group is responsible for the alignment and calibration of each detector.
    The shifts and rotations necessary for alignment are included in the Calibration and Conditions Database (CCDB), which is then utilised during reconstruction.

    The alignment work aimed to achieve three primary goals.
    Firstly, to provide FMT alignment tables to Run Group F (RG-F) for use in reconstruction.
    Secondly, to assess whether the resolution improvement obtained from the FMT justifies the additional material introduced into the CLAS12 detector.
    Lastly, to offer detailed information about these improvements, enabling Run Group E (RG-E) and other run groups to make informed decisions on whether to include the detector in their runs.

% --+ Definitions +-------------------------------------------------------------
    Alignment shifts can be performed in any of the three global axes: $z$, which is aligned with the beamline; $x$, which runs parallel to the ground; and $y$, which points upwards from the ground.
    Additionally, alignment rotations can be carried out around these axes.
    Specifically, for the purposes of this work, rotations around the $z$ axis are referred to as $\phi$ rotation (roll), while rotations around the $x$ and $y$ axes are termed pitch and yaw, respectively.

    To quantify misalignment, the DOCA between a reconstructed DC track and an FMT cluster is defined as a Residual.
    Due to the geometry of each layer (as depicted in Figure \ref{fig::12.10::fmt_geometry}), only the residuals in the local $y$ axis of a layer (perpendicular to the strips) can be measured.
    This implies that global $z$ and $\phi$ alignment can be performed independently for each layer.
    However, global $x$, global $y$, pitch, and yaw alignment must be carried out simultaneously for the entire detector.

% --+ How was it done +---------------------------------------------------------
    \begin{figure}[b!]
        \includegraphics[width=\textwidth]{20res_example.png}
        \caption[Example FMT residuals plot]
        {Example Forward Micromegas Tracker (FMT) residuals plot.}
        \floatfoot{Source: Own elaboration, using \href{https://github.com/JeffersonLab/clas12alignment}{CLAS12 alignment software}.}
        \label{fig::12.20::fmt_residuals_example}
    \end{figure}

    To minimise residuals, they are plotted for a specific shift or rotation in the relevant axes.
    An example of such a plot is illustrated in Figure \ref{fig::12.20::fmt_residuals_example}.
    Since the residuals are expected to follow a Gaussian distribution, a Gaussian fit is applied to them.

    For $z$ and $\phi$ alignment, the goodness of fit is heuristically evaluated by comparing the standard deviation ($\sigma$) of the Gaussian fits and selecting the shift with the smallest $\sigma$.
    For $x$, $y$, pitch, and yaw alignment, the goodness of fit is heuristically evaluated by choosing the fit with the mean closest to zero.
    It is important to consider a reasonable error margin when selecting the minima.

    Examples of the distributions of goodness of fit for $z$ and $xy$ alignment can be observed in Figure \ref{fig::12.20::fmt_residuals_fit_example}.

    \begin{figure}[t!]
        \includegraphics[width=\textwidth]{20resfit_example.png}
        \caption[Examples of residuals goodness of fit plots]
        {Examples of residuals goodness of fit plots.}
        \floatfoot{Source: Own elaboration, using \href{https://github.com/JeffersonLab/clas12alignment}{CLAS12 alignment software}.}
        \label{fig::12.20::fmt_residuals_fit_example}
    \end{figure}

    % !TEX root = ../main.tex
\subsubsection{Fiducial Cuts}
\label{sssec::fiducial_cuts}
    To reduce background, fiducial cuts are applied to the DC tracks and FMT clusters.
    This process is useful to increase data quality in order to obtain meaningful alignment results.

    For DC tracks, the cuts applied are:
    \begin{itemize}
        \item $\text{track}.z < \text{layer}.z$:
        Remove tracks with a vertex $z$ further downstream than the FMT layer before swimming.
        This is caused by reconstruction errors where the particle origin is outside of the target.
        % $9.84\%$ of tracks fail to meet this criterion in the sample data.
        \item $\mid\text{track}.z - \text{layer}.z\mid < 0.05 \text{cm}$:
        Remove tracks too far from the FMT layer after swimming.
        This was caused by bugs in the swimmer which will be mentioned in the next section.
        % $18.67\%$ tracks failed to meet this criterion.
        \item $5 \text{cm} < \sqrt{x^2 + y^2} < 25 \text{cm}$:
        Remove tracks outside of the layer's active region.
        % $35.22\%$ of the tracks were lost to this criterion.
        \item $\theta < ~66.42^{\circ}$:
        Remove tracks with a $\theta$ angle too high.
        When this happens, the same particle is affecting many strips, which causes the detector's data to not be as reliable as we want for alignment.
        % $7.22\%$ of tracks were lost to this criterion.
    \end{itemize}
    % From all these criteria, $70.95\%$ of the DC tracks were lost.
    % It is worth noting that after some reconstruction errors were fixed (as will be detailed in the following section), this percentage was reduced to $52.28\%$.

    For FMT clusters, the cuts applied are:
    \begin{itemize}
        \item $50 \text{ns} < \text{T}_{\text{min}} < 500 \text{ns}$:
        Cut clusters with an illogical $\text{T}_{\text{min}}$, which is the time of the first hit in the cluster.
        % $21.12\%$ of clusters fail to meet this criterion.
        \item $\text{size} > 1$ $\&\&$ $\text{E} > 100$:
        Cut small clusters with high energy, which are generally considered bad.
        % Only $7.36\%$ clusters are lost to this criterion.
        \item $\text{size} < 5$:
        Cut large clusters, which are not considered very useful.
        % Only $7.77\%$ are lost to this criterion.
    \end{itemize}
    % From all these criteria, $36.25\%$ of the FMT clusters were lost.

    % !TEX root = ../main.tex
\subsubsection{Residuals Improvements}
\label{12.22::residuals_improvements}
    To validate the proposed alignment algorithm, it was applied to the data from Run Group F (RG-F), specifically Run 11983.
    The improvement in residuals is readily apparent when comparing the before and after alignment results, as depicted in figure \ref{fig::12.22::fmt_residuals_comparison}.
    It is important to note the difference in scale between the top and bottom plots, which further highlights the significant improvement achieved through the alignment process.

    \begin{figure}[t!]
        \centering\frame{
        \includegraphics[width=\textwidth]{22res_comparison.png}}
        \caption[Residuals distribution improvement.]{Residuals distribution before (upper image) and after (lower image) alignment.
        Source: \hyperlink{github.com/JeffersonLab/clas12alignment}{CLAS12 alignment software}.}
        \label{fig::12.22::fmt_residuals_comparison}
    \end{figure}

    As depicted in the figure, the $z$ and $\phi$ alignment significantly reduces the background, resulting in a higher concentration of residuals around the mean of the distribution.
    Moreover, the $x$ and $y$ alignment effectively aligns the mean of the distribution closer to zero, improving the overall alignment.
    However, for the pitch and yaw alignment, meaningful results could not be obtained.
    This can be attributed to the limited data available from the three layers, combined with the small rotations around the $x$ and $y$ axes, making it challenging to achieve precise alignment.

    To determine the mean and standard deviation ($\sigma$) of the distribution, a Gaussian fit was applied. The parameters of the Gaussian fit are
     \begin{align*}
        \Big( \text{amp} \cdot \text{gaus}(\mu, \sigma) \Big) &+ \Big( p_0 + p_1\cdot x + p_2\cdot x^2 \Big) \\
        \text{gaussian} \hspace{0.8cm} &+ \hspace{1cm} \text{background}
    \end{align*}

    The results obtained are included in the CCDB at:

    \small\href{clasweb.jlab.org/cgi-bin/ccdb/versions?table=/geometry/fmt/alignment}{\texttt{clasweb.jlab.org/cgi-bin/ccdb/versions?table=/geometry/fmt/alignment}}

    Alignment was successfully performed for the data from Run Group M (RG-M), demonstrating that the alignment procedure is not specific to a particular run.
    This indicates the general applicability and effectiveness of the alignment procedure across different runs in the CLAS12 detector.

    The impact of this alignment procedure on the resolution of the entire CLAS12 detector will be further investigated and discussed in the concluding subsection of the section.


    The procedure described in this section is documented and shared publicly.
    It can be seen at

    \href{https://github.com/JeffersonLab/clas12alignment/tree/master/fmt}{\texttt{github.com/JeffersonLab/clas12alignment/tree/master/fmt}}
