% !TEX root = ../main.tex
\subsubsection{Fiducial Cuts}
\label{12.21::fiducial_cuts}
    To mitigate background noise, fiducial cuts are applied to the DC tracks and FMT clusters.
    This process enhances data quality, ensuring more meaningful alignment results.

    For DC tracks, the following cuts are implemented:
    \begin{itemize}
        \item
            $\text{track}.z < \text{layer}.z$:
            This cut removes tracks with a vertex $z$ position further downstream than the FMT layer prior to swimming.
            Such occurrences result from reconstruction errors where the particle origin is outside the target.
        \item
            $\mid\text{track}.z - \text{layer}.z\mid < 0.05 \text{cm}$:
            This eliminates tracks that are too far from the FMT layer after swimming, caused by bugs in the swimmer process that will be discussed in the subsequent section.
        \item
            $5 \text{cm} < \sqrt{x^2 + y^2} < 25 \text{cm}$:
            This removes tracks outside the active region of the layer.
        \item
            $\theta < ~66.42^{\circ}$:
            This excludes tracks with excessively high $\theta$ angles.
            When this occurs, a single particle affects multiple strips, rendering the detector's data less reliable for alignment purposes.
    \end{itemize}

    The implemented fiducial cuts applied to the FMT clusters are:
    \begin{itemize}
        \item
            $50 \text{ns} < \text{T}{\text{min}} < 500 \text{ns}$:
            This cut removes clusters with illogical values for $\text{T}{\text{min}}$, which represents the time of the first hit in the cluster.
        \item
            $\text{size} > 1$ $\&$ $\text{E} > 100$:
            This eliminates small clusters with high energy, as they are generally considered to be of poor quality.
        \item
            $\text{size} < 5$:
            This discards large clusters, as they are deemed to be less useful for analysis purposes.
    \end{itemize}
