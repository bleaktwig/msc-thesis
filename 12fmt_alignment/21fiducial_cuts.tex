% !TEX root = ../main.tex
\subsubsection{Fiducial Cuts}
\label{sssec::fiducial_cuts}
    To reduce background, fiducial cuts are applied to the DC tracks and FMT clusters.
    This process is useful to increase data quality in order to obtain meaningful alignment results.

    For DC tracks, the cuts applied are:
    \begin{itemize}
        \item $\text{track}.z < \text{layer}.z$:
        Remove tracks with a vertex $z$ further downstream than the FMT layer before swimming.
        This is caused by reconstruction errors where the particle origin is outside of the target.
        % $9.84\%$ of tracks fail to meet this criterion in the sample data.
        \item $\mid\text{track}.z - \text{layer}.z\mid < 0.05 \text{cm}$:
        Remove tracks too far from the FMT layer after swimming.
        This was caused by bugs in the swimmer which will be mentioned in the next section.
        % $18.67\%$ tracks failed to meet this criterion.
        \item $5 \text{cm} < \sqrt{x^2 + y^2} < 25 \text{cm}$:
        Remove tracks outside of the layer's active region.
        % $35.22\%$ of the tracks were lost to this criterion.
        \item $\theta < ~66.42^{\circ}$:
        Remove tracks with a $\theta$ angle too high.
        When this happens, the same particle is affecting many strips, which causes the detector's data to not be as reliable as we want for alignment.
        % $7.22\%$ of tracks were lost to this criterion.
    \end{itemize}
    % From all these criteria, $70.95\%$ of the DC tracks were lost.
    % It is worth noting that after some reconstruction errors were fixed (as will be detailed in the following section), this percentage was reduced to $52.28\%$.

    For FMT clusters, the cuts applied are:
    \begin{itemize}
        \item $50 \text{ns} < \text{T}_{\text{min}} < 500 \text{ns}$:
        Cut clusters with an illogical $\text{T}_{\text{min}}$, which is the time of the first hit in the cluster.
        % $21.12\%$ of clusters fail to meet this criterion.
        \item $\text{size} > 1$ $\&\&$ $\text{E} > 100$:
        Cut small clusters with high energy, which are generally considered bad.
        % Only $7.36\%$ clusters are lost to this criterion.
        \item $\text{size} < 5$:
        Cut large clusters, which are not considered very useful.
        % Only $7.77\%$ are lost to this criterion.
    \end{itemize}
    % From all these criteria, $36.25\%$ of the FMT clusters were lost.
