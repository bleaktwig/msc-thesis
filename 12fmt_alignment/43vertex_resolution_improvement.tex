% !TEX root = ../main.tex
\subsubsection{Vertex Resolution Improvement}
\label{12.43::vertex_resolution_improvement}
    Furthermore, an additional cut has been introduced.
    As of the FMT alignment work, beam alignment for RG-F data had not been carried out, resulting in a decrease in vertex resolution.
    To mitigate the impact of this alignment issue on reconstruction accuracy, we have implemented a cut to utilise only one sector of the detector.

    \begin{figure}[b!]
        \centering\frame{
        \includegraphics[scale=0.24]{40dc_vs_fmt_sector1.png}}
        \caption[DC vs FMT $z$ with geometry correction]{DC vs FMT vertex $z$ for electrons with a geometric correction. DC tracks are shown in green while FMT tracks are shown in blue. Data from only one CLAS12 sector was used to obtain this plot.
        Source: \texttt{fmtVertex.groovy} script in \hyperlink{github.com/JeffersonLab/clas12alignment}{CLAS12 alignment software}.}
        \label{fig::12.43::dc_vs_fmt_vz_11983_corrected}
    \end{figure}

    The resolution plot, comparing DC and FMT tracks after applying all the previously mentioned cuts, is depicted in Figure \ref{fig::12.43::dc_vs_fmt_vz_11983_corrected}.
    To evaluate the resolution for both DC and FMT tracks, we utilise a fit consisting of two Gaussian curves combined with a quadratic curve to account for the background.
    The fit is defined as follows
    \begin{equation*}
        \text{amp}_1 \cdot \text{gaus}(z, z_\text{max}, \sigma) + \text{amp}_2 \cdot \text{gaus}(z, z_\text{max} - 2.4, \sigma) + p_1 + p_2\cdot z + p_3\cdot z^2,
    \end{equation*}
    where
    \begin{itemize}
        \item
            $\text{amp}1$ represents the amplitude of the largest peak, and $z\text{max}$ corresponds to its $z$ position,
        \item
            $\text{amp}_2$ signifies the amplitude of the leftward peak, which has been measured to be at a position of $2.4$ centimetres, and
        \item
            the remaining parameters, $p_1$, $p_2$, and $p_3$, are obtained through the fitting process.
    \end{itemize}

% --+ Resolution for electrons +------------------------------------------------
    For electrons in run 011983 (low luminosity, 50 nA), the analysis yields a DC resolution of $\sigma_\text{DC} = 0.875$ cm and an FMT resolution of $\sigma_\text{FMT} = 0.387$ cm.
    This indicates a doubling of the resolution achieved with the inclusion of the FMT detector.

    Similarly, for electrons in run 012016 (production luminosity, $250$ nA), the analysis shows a DC resolution of $\sigma_\text{DC} = 1.009$ cm and an FMT resolution of $\sigma_\text{FMT} = 0.596$ cm.
