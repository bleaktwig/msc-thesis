% !TEX root = ../main.tex
\subsection{FMT Reconstruction Work}
\label{12.30::fmt_reconstruction_work}
% --+ Why is it needed +--------------------------------------------------------
    As the work on FMT alignment progressed, certain modifications were required in FMT reconstruction.
    These modifications primarily involved incorporating the alignment shifts determined during the alignment process into the reconstruction.
    Additionally, some fixes were made to address issues that were identified during the alignment work.
    These changes ensure that the reconstruction process takes into account the alignment information and addresses any related issues that were encountered.

% --+ What was done +-----------------------------------------------------------
    First, the loading of shifts from the CCDB was included in the standard geometry class of the FMT reconstruction package.
    Then, standard methods to include the shifts in any frame of reference change were implemented.
    Finally, the code was studied in detail, and the shifts were added in all instances where they were required since the package originally didn't consider their application.

    ``Crossmaking'' is the process of matching clusters in different layers to obtain an accurate 3D estimate of the position of a track \cite{ziegler2020}.
    This process was initially included in FMT reconstruction to facilitate the reconstruction for the six FMT layers.
    However, as mentioned before, only three FMT layers were installed for the RG-F run, and future runs may also use three layers due to concerns with the Lorentz angle when using six layers.
    To simplify the reconstruction process and make better use of the available number of layers, crossmaking was removed from the reconstruction.

    Outside of FMT reconstruction, some minor changes were also required in the DC reconstruction package since some of its components depend on the FMT layers' positions.
    Additionally, the swim package diagnostic was updated as it failed to properly reconstruct the positions of tracks near the FMT layers.

% --+ Juicy link +--------------------------------------------------------------
    A detailed list of the updates applied can be found in the following pull request to the \texttt{clas12-offline-software} repository:

    \href{github.com/JeffersonLab/clas12-offline-software/pull/726}{\texttt{github.com/JeffersonLab/clas12-offline-software/pull/726}}.
