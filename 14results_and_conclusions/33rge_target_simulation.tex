% !TEX root = ../main.tex
\subsubsection{RG-E Target Simulation}
\label{14.33::rge_target_simulation}
% --+ Setup +-------------------------------------------------------------------
    As a way to verify the validity of the results, we determined to run a simulation with the RG-E target placed in the selected location.
    The methology for this simulation is the same as the one used with the RG-F target explained in Section \ref{13.40::acceptance_correction}.

    First, 10 million events were generated in deep inelastic kinematics using LEPTO \cite{ingelman1997}.
    Half of these were done with a D2 target, and the other half with a C target.
    The $v_z$ of the D2 kinematics were randomised considering the 3 cm long liquid target, placed from $-5$ to $-2$ cm.
    The $v_z$ of the C kinematics were placed at 2 cm considering a negligible target width.
    These positions are based on the results obtained in Sections \ref{14.31::phase_space_study} and \ref{14.32::statistics_study}.

    Subsequently, the events were simulated in the same experimental conditions as those for the RG-F experiment in CLAS12 using \texttt{gemc} \cite{ungaro2020gemc}.
    The simulation was set with a beam energy of 11 GeV, a torus field polarity of $-1$ and a solenoid field polarity of $-0.745033$.
    The events were then reconstructing using \texttt{coatjava} \cite{ziegler2020}.

% --+ TODO. Results +-----------------------------------------------------------
    A $v_z$ plot of the targets is presented in Figure \textbf{TODO}, where we compare the DC and FMT vertex position.
    As can be seen on the plot, the positions of both the D2 and the C target were correctly simulated.
    Additionally, setting the separation between the two targets to 4 cm helps maintain the clarity of the two peaks.
    As was noted in Section \ref{12.44::conclusions}, this is one of the benefits of FMT: Its improved vertex resolution allows for double targets to be placed closer to each other, benefiting the physics analysis.

    Integrated DIS plots are presented in Figures \textbf{TODO} to \textbf{TODO}.
    Naturally, these results are not acceptance-corrected, as they are from simulated data.
    As can be seen on the plots, the phase space of each DIS variable is the maximum possible.
    This validates the results found in Section \ref{14.31::phase_space_study}.
    % TODO. Note any particularities in any of the plots for sweet sweet extra content.

    Finally, the acceptance plots for the DIS variables are presented in Figures \textbf{TODO} and \textbf{TODO}.
    For the $e^-$ variables, we can compare Figure \textbf{TODO} with that from the RG-F target simulation, in Figure \ref{fig::14.21::electron_acc}.
    % TODO. Draw conclusions.

    Then, for the hadronic variables, we compare Figure \textbf{TODO} with Figure \ref{fig::14.22::hadronic_acc}.
    % TODO. Draw conclusions.

    As a final note, all this analysis pertains to simulated data.
    The final validation of the results in this document will come from the actual execution of the RG-E experiment.

% --+ TODO. Plots +-------------------------------------------------------------
