% !TEX root = ../main.tex
\subsubsection{RG-E Target Simulation}
\label{14.33::rge_target_simulation}
% --+ Setup +-------------------------------------------------------------------
    As a way to verify the validity of the results, we determined to run a simulation with the RG-E target placed in the selected location.
    The methology for this simulation is the same as the one used with the RG-F target explained in Section \ref{13.40::acceptance_correction}.

    10 million events were generated in deep inelastic kinematics using LEPTO \cite{ingelman1997}.
    Half of these were done with a D2 target, and the other half with the C target.
    The $v_z$ of the D2 kinematics were randomised considering the 3 cm long liquid target, placed from $-5$ to $-2$ cm.
    The $v_z$ of the C kinematics were placed at 2 cm considering a negligible target width.
    These positions are based on the results obtained in Sections \ref{14.31::phase_space_study} and \ref{14.32::statistics_study}.

    Subsequently, the events were simulated in the same experimental conditions as those for the RG-F experiment in CLAS12 using \texttt{gemc} \cite{ungaro2020gemc}.
    Again, the simulation was set with a torus field polarity of $-1$ and a solenoid field polarity of $-0.745033$.
    The events were then reconstructing using \texttt{coatjava} \cite{ziegler2020}.

% --+ TODO. Results +-----------------------------------------------------------

% --+ TODO. Plots +-------------------------------------------------------------
