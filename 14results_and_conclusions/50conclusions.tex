% !TEX root = ../main.tex
\subsection{Conclusions}
\label{14.50::conclusions}
    This thesis consisted of three subjects: the alignment of the FMT detector, the development of a standard analysis software for RG-E, and the selection of a location for the RG-E target.

% --+ FMT Alignment +-----------------------------------------------------------
    Regarding the FMT alignment, we developed the standard FMT alignment software, which works by minimising the track residuals.
    Certain modifications to the FMT reconstruction software were also developed to integrate the alignment.
    This work resulted in an $e^-$ $z$ resolution of $\sigma_\text{FMT} = 0.387$ cm for low luminosity ($50$ nA) runs and $\sigma_\text{FMT} = 0.596$ cm for high luminosity ($250$ nA) runs.
    A detailed rundown of the subject can be read in Section \ref{12::fmt_alignment_and_reconstruction}.

% --+ Analysis Software +-------------------------------------------------------
    Regarding the standard analysis software, the entire toolset was developed as part of the work related to this thesis.
    The software works as expected.
    As evidence of this, all the analyses done in Sections \ref{13.30::sampling_fraction}, \ref{13.40::acceptance_correction}, and \ref{14::results_and_conclusions} were made using this same code.
    Most of the figures presented in this thesis were obtained using this software, as shown in their sources.
    Additionally, other preparatory analyses for the RG-E experiment have used the software to obtain their results.
    The software itself is explained in Section \ref{13.10::clas12_rge_analysis}.

% --+ Target Location +---------------------------------------------------------
    Using the code just described, we selected the best location for the RG-E target based on phase space and statistics studies.
    For the phase space study, the range of two electron variables -- $Q^2$ and $\nu$ -- and three hadronic variables -- $z_h$, $p_T^2$, and $\phi_{PQ}$ -- was studied for ten 5 cm bins in $v_z$.
    A range from $-5$ to $10$ cm was obtained from this study, based on maximising the phase spaces.
    Then, for the statistic study, a 7 cm region was chosen in this range based on maximising the number of events.
    The region chosen was from $-5$ to $2$ cm.
    % A validation of the region was then performed by running a simulation of the RG-E target placed in it.
    % This analysis is detailed in Section \ref{14.30::study_results}.

% --+ Sampling Fraction & Acceptance Correction Procedures +--------------------
    In order to improve the quality of the results, procedures for sampling fraction estimation and detector acceptance correction were proposed and applied.
    Both were done successfully, and the statistical error from the acceptance correction process was summarily propagated to the results.
    The application of the former is detailed in Section \ref{13.30::sampling_fraction}, and that of the latter in Sections \ref{13.40::acceptance_correction} and \ref{14.20::acceptance_correction_results}.

% --+ Efficiency Studies +------------------------------------------------------
    In addition to the described analyses, a detector efficiency study was conducted on the FMT.
    The three main factors in this efficiency were found and studied: one is related to the detector alignment, one to its geometry, and one to its offline reconstruction.
    A geometry cut was derived from the second, contributing to the accuracy of the entire study presented in this document.
    The process and results are described in Section \ref{14.10::fmt_efficiency}.
