% !TEX root = ../main.tex
\subsubsection{Phase Space Study}
\label{14.31::phase_space_study}
    % Introduction.
    Considering the objectives of this DIS study, it is advantageous to maximise the phase space of each kinematic variable of study.
    This approach broadens the scope of investigation, increases sensitivity to detect rare phenomena, and facilitates the testing of theoretical predictions for future studies using the double target system.
    Therefore, the first criterion for selecting a $v_z$ region for the target is to find a region that provides the maximum range of kinematic variables.

    % Resulting plots.
    The resulting plots show the acceptance-corrected DIS variables separated into $v_z$ bins.
    Figures \ref{fig::14.31::q2_vz} and \ref{fig::14.31::nu_vz} display the distributions of the electron variables $Q^2$ and $\nu$, respectively.
    The $z_h$ distributions for $e^-\pi^+$ and $e^-\pi^-$ can be observed in Figures \ref{fig::14.31::zh_211_vz} and \ref{fig::14.31::zh_-211_vz}, respectively.
    Figures \ref{fig::14.31::pt2_211_vz} and \ref{fig::14.31::pt2_-211_vz} show the distributions of $p_T^2$ for $e^-\pi^+$ and $e^-\pi^-$, respectively.
    Finally, Figures \ref{fig::14.31::phipq_211_vz} and \ref{fig::14.31::phipq_-211_vz} present the distributions of $\phi_{PQ}$ for $e^-\pi^+$ and $e^-\pi^-$, respectively.
    These plots provide insights into the dependence of each DIS variable on the $v_z$ coordinate.
    For these same distributions without acceptance correction, please see Appendix \ref{20.04::dis_vz_plots}.

    % Q2.
    In the study of $Q^2$, as shown in Figure \ref{fig::14.31::q2_vz}, the higher end of the variable's phase space is limited for $v_z < -5$ cm, with the effect becoming more pronounced as we move further upstream.
    This effect can be understood by considering the compounded effect of the $\theta$ efficiency for negative particles (as seen in Figure \ref{fig::14.21::theta_study_neg}) and the limited acceptance region of FMT (described by Equation \eqref{eq::12.42::fmt_geometry_cut} and illustrated in Figure \ref{eq::12.42::vz_vs_theta}).

    The higher end of $\theta$ becomes limited for lower $v_z$ values.
    Based on the objective of maximising the phase space of each variable, this suggests setting the minimum $v_z$ for the RG-E target near $-5$ cm.
    Additionally, it is noted that the variable exhibits an unusual shape for $10$ cm $< v_z < 20$ cm, likely due to the cut in low $\theta$ angles in that region, which is another consequence of the FMT acceptance region.

    % nu.
    In the study of $\nu$, as seen in Figure \ref{fig::14.31::nu_vz}, it was previously observed that $\nu$ has no direct correlation with the scattering angle $\theta_C$ (Section \ref{14.20::acceptance_correction_results}).
    Therefore, no significant effect on the phase space of $\nu$ is observed for $v_z < -5$ cm, unlike $Q^2$.
    However, a loss is observed in the lower end of the phase space for $v_z = 10$ cm and downstream.
    Based on this effect, it is reasonable to keep $v_z$ below approximately 10 cm to preserve the largest possible phase space of $\nu$.

    % zh.
    In the study of $z_h$, despite its lack of direct correlation with the electron's and pion's $\theta$, clear differences are observed across different $v_z$ bins, as shown in Figures \ref{fig::14.31::zh_211_vz} and \ref{fig::14.31::zh_-211_vz}.
    However, this can be explained by its inverse correlation with $\nu$ (as described in Equation \eqref{eq::10.32::zh}).
    Similar to $\nu$, the extreme phase space loss is primarily observed for $v_z > 10$ cm, and therefore, no additional severe restrictions on the $v_z$ region are imposed beyond those defined based on the studies of $Q^2$ and $\nu$.

    % pt2.
    In the study of $p_T^2$, as depicted in Figures \ref{fig::14.31::pt2_211_vz} and \ref{fig::14.31::pt2_-211_vz}, large statistical fluctuations are observed for $p_T^2 > 1.4 \text{GeV}^2$, consistent with the prediction in Section \ref{14.22::hadronic_variables}.
    Studying the phase space of the variable, a cutoff at high $p_T^2$ values is observed for $v_z < -5$ cm and $v_z > 15$ cm, similar to what was seen for $Q^2$.
    Based on this observation, no additional restrictions are imposed on the $v_z$ region under study.

    % phipq.
    Regarding the study of $\phi_{PQ}$, as shown in Figures \ref{fig::14.31::phipq_211_vz} and \ref{fig::14.31::phipq_-211_vz}, no easily discernible loss is observed in the phase space of $\phi_{PQ}$ as we vary $v_z$.
    While there are significant changes in the shape of the variable distribution across different $v_z$ bins, conducting a detailed shape study is beyond the scope of this thesis, as ample information is already provided by the other DIS variables.
