% !TEX root = ../main.tex
\subsection{Study Results}
\label{14.30::study_results}


    % statistical error estimation.
    The total statistical error on the acceptance corrected result $e_\text{corr}$ needs to consider both the statistical error of the measurements $e_\text{meas}$ and that of the acceptance correction $e_\text{acc}$.
    The former is purely statistical in nature, and is thus given by
    \begin{equation*}
        e_\text{meas} = \frac{\delta y_\text{meas}}{y_\text{meas}},
    \end{equation*}
    and the latter was derived in Equation \eqref{eq::14.20::acc_error}.

    Considering the fact that $e_\text{meas}$ comes purely from experimental data and $e_\text{acc}$ comes purely from simulation, they are completely uncorrelated.
    Thus, we can estimate the total statistical error of the acceptance corrected result as the quadratic addition of the two, or
    \begin{equation*}
        e_\text{corr} = \sqrt{e_\text{meas}^2 + e_\text{acc}^2}.
    \end{equation*}

    % TODO. systematic error "estimation".
    %   * TODO. Ask Raffaella for a reference about the "average" systematic error we should consider.

    % \input{14results_and_conclusions/31integrated_dis}
    % TODO. DIS variables over all the region.

    % TODO. First criterium for selecting a good bin is maximising the phase space of each variable.
    % TODO. Then, the second criterium is maximising the statistics of that variable.
        % NOTE. As Hayk if I should use the results without acceptance correction to deduce the statistics?

    % \input{14results_and_conclusions/32binned_dis}
    % TODO. DIS variables separated over vz.

% TODO. Fits. Talk about fits.
