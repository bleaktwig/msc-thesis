% !TEX root = ../main.tex
\subsection{Conclusions}
    \label{14.40::conclusions}
    This thesis consisted of three subjects: the alignment of the FMT detector, the development of a standard analysis software for RG-E, and the selection of a location for the RG-E target.

% --+ FMT Alignment +-----------------------------------------------------------
    About FMT alignment, we developed the standard FMT alignment software which works by minimising the track residuals.
    We also developed certain modifications to the FMT reconstruction software to integrate the alignment.
    This work resulted in an $e^-$ $z$ resolution of $\sigma_\text{FMT} = 0.387$ cm for low luminosity ($50$ nA) runs, and one of $\sigma_\text{FMT} = 0.596$ cm for high luminosity ($250$ nA) runs.
    A detail rundown of the subject can be read in Section \ref{12::fmt_alignment_and_reconstruction}.

% --+ Analysis Software +-------------------------------------------------------
    Regarding the standard analysis software, the entire toolset was developed as part of the work related to this thesis.
    The software works as expected.
    As evidence of this, all the analysis done in Sections \ref{13.30::sampling_fraction}, \ref{13.40::acceptance_correction}, and \ref{14::results_and_conclusions} was made using this same software.
    Most of the Figures presented on this thesis were obtained using this software, as is shown on their sources.
    Additionally, other preparatory analyses for the RG-E experiment have used the software to obtain their results.
    The software itself is explained in Section \ref{13.10::clas12_rge_analysis}.

% --+ Target Location +---------------------------------------------------------
    Using the software just described, we selected the best location for the RG-E target based on phase space and statistics studies.
    For the phase space study, the range of two electron variables -- $Q^2$ and $\nu$ -- and three hadronic variables -- $z_h$, $p_T^2$, and $\phi_{PQ}$ -- was studied for ten 5 cm bins in $v_z$.
    A range from $-5$ to $10$ cm was obtained from this study, based on maximising the phase spaces.
    Then, for the statistic study, a 7 cm region was picked in this range based on maximising the number of events.
    The region chosen was from $-5$ to $2$ cm.
    A validation of the region was then made by running a simulation of the RG-E target placed in it.
    This analysis is detailed in Section \ref{14.30::study_results}.

% --+ Sampling Fraction & Acceptance Correction Procedures +--------------------
    In order to improve the quality of the results, procedures for sampling fraction estimation and detector acceptance correction were proposed and applied.
    Both were done successfuly, and the statistical error from the acceptance correction process was summarily propagated to the results.
    The application of the former is detailed in Section \ref{13.30::sampling_fraction}, and of the latter in Sections \ref{13.40::acceptance_correction} and \ref{14.20::acceptance_correction_results}.

% --+ Efficiency Studies +------------------------------------------------------
    In addition to the described analyses, a detector efficiency study was done to FMT.
    The three main factors in this efficiency were found and studied: one is related to the detector alignment, one to its geometry, and one to its offline reconstruction.
    A geometry cut was derived from the second, helping the accuracy of the entire study presented in this document.
    The process and results is described in Section \ref{14.10::fmt_efficiency}.

% --+ Unestimated Systematic Errors +-------------------------------------------
    % TODO. This grew too much. Should be moved to its own subsection.
    As the attentive reader will notice, no systematic errors were applied to the results of this document.
    This is only due to the broadness of the analysis, and a more detailed one would of course need to include them.
    Based on the list provided in \cite{osipenko2010}, the source of systematic uncertainties we estimate would be relevant to this study are the following:
    \begin{itemize}
        \item
            Previous CLAS DIS measurements \cite{osipenko2006} on deuterium targets showed that the combined efficiency-acceptance systematic uncertainty of CLAS data averages to $4.5\%$.
            We must be careful when extrapolating the results of this study to CLAS12 data, but the error dependence on $\nu/E_b$ allows us to estimate that the higher beam energy of CLAS12 would reduce this uncertainty.

        \item
            As was mentioned in Section \ref{11.230::offline_reconstruction}, particle misidentification can be a significant source of systematic error.
            Based on the results presented in Table \ref{tab::11.232::reconstruction_pid}, we assume that the $e^-$ misidentification uncertainty would be minor, while that for $\pi^\pm$ would need to be studied with closer scrutiny.
            For the estimation of this systematic error, we would need to study the particle identification scheme as described in Section \ref{11.230::offline_reconstruction}, and understand the accuracy of each detector involved.

        \item
            No radiative corrections were applied in this document, so we can assume that a significant systematic error related to radiative effects is left unestimated in the analysis.
            We hope that the radiative cut described in Section \ref{13.23::dis_cuts} ($Y_b < 0.85$) addresses most of this uncertainty.

        \item
            TODO. Bin migration.

        \item
            Systematic uncertainties can arise in the Monte Carlo simulation described in Section \ref{13.40::acceptance_correction}.
            To account for these, we would need to study the precision of LEPTO in predicting the cross sections relevant to DIS.
            Then, we would need to study how well GEMC accounts for the CLAS12 response to these cross sections, based on publications such as \cite{ungaro2020gemc}.

        \item
            TODO. e- momentum correction through sampling fraction.
            There is an uncertainty associated to the process of momentum correction, such as the one done for $e^-$ via sampling fraction, described in Section \ref{13.30::sampling_fraction}.
            Based on the methodology and formulae described in that same section, we could derive a function to estimate the momentum-dependent systematic error inherent to the method.
    \end{itemize}

    After studying all these effects in detail, we would need to sum them in quadrature.
    We hope to be able to do this study for future analyses that come from this document.
