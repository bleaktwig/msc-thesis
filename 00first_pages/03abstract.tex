% !TEX root = ../main.tex
\section*{Abstract}
    \noindent \textbf{Abstract---}
        Particle accelerators and detectors are the main source of data for High Energy Physics.
        In this context, they are usually massive machines with a great number of moving parts, most of which require work in calibration and maintenance to run in optimal conditions.
        Special care must be put into the software dedicated to this calibration, in addition to event reconstruction.

        This thesis presents a calibration effort of this kind, done to the newest detector in the CLAS12 reconstruction chain --- the Micromegas Vertex Tracker.
        The results obtained in this regard are favourable.
        The calibration proved successful, doubling vertex resolution, and providing harsher criteria for selecting useful particle tracks.

        In addition, the thesis goes into detail about the improvements in reconstruction that come from this new detector, focusing on its acceptance, vertex resolution, and SIDIS variables.
        A full study is presented for the total acceptance in CLAS12, and the multiplicities of various types of particles are measured.

    \noindent \textbf{Keywords---}
        Particle detectors; Jefferson Laboratories; CLAS12; Micromegas Detectors.

    \vspace{1.0cm}

    \noindent \textbf{Resumen---}
        Los aceleradores y detectores de partículas son la principal fuente de datos para la Física de Altas Energías.
        En este contexto, estos suelen ser máquinas masivas con un gran número de partes móviles, la mayoría de las cuales requieren un trabajo de calibración y mantenimiento para funcionar en condiciones óptimas.
        Hay que tener especial cuidado en el \textit{software} dedicado a esta calibración, además del dedicado a la reconstrucción de eventos.

        Esta tesis presenta un trabajo en calibración de este tipo, realizado al detector más nuevo de la cadena de reconstrucción de CLAS12 --- el \textit{Micromegas Vertex Tracker}.
        Los resultados obtenidos son favorables.
        La calibración resultó exitosa, duplicando la resolución de vértice y proporcionando criterios más duros para la selección de \textit{tracks} de partículas útiles.

        Además, la tesis profundiza en las mejoras en reconstrucción que aporta este nuevo detector, centrándose en su \textit{acceptance}, la resolución de vértices y las variables de SIDIS.
        Se presenta un estudio completo de \textit{acceptance} total de CLAS12, y se miden las multiplicidades de varios tipos de partículas.

    \noindent \textbf{Palabras Clave---}
        Detectores de partículas; Jefferson Laboratories; CLAS12; Detectores Micromegas.
