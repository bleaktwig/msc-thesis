% !TEX root = ../main.tex
\subsection{Acceptance Correction}
\label{ssec::acceptance_correction}
% --+ What is acceptance +------------------------------------------------------
    When discussing radiation detection, it is customary to distinguish between two types of efficiency: absolute efficiency and intrinsic detection efficiency.
    The former is defined as the fraction of events emitted by the source that are actually detected by the detector, expressed as
    \begin{equation*}
        \xi_\text{tot} = \frac{\text{events registered}}{\text{events emitted by source}}.
    \end{equation*}

    This efficiency is influenced by the detector's geometry and the probability of an interaction occurring within the detector.
    The total efficiency is also referred to as the detector acceptance.

    The total efficiency can be further decomposed into two components: the intrinsic efficiency, $\xi_{\text{int}}$, and the geometric efficiency, $\xi_{\text{geom}}$.
    The total efficiency is then given by
    \begin{equation*}
        \xi_\text{tot} = \xi_\text{int} \cdot \xi_\text{geom}.
    \end{equation*}

    The intrinsic efficiency represents the fraction of events that actually reach and are detected by the detector
    \begin{equation*}
        \xi_\text{int} = \frac{\text{events registered}}{\text{events impinging on detector}}.
    \end{equation*}

    This probability is dependent on the interaction cross-sections of the incident radiation with the detector medium.
    The intrinsic efficiency thus varies with the type of radiation, its energy, and the detector material \cite{leo1987}.

% --+ Acceptance correction through generation + simulation +-------------------
    Acceptance correction involves compensating for the total efficiency of the detector.
    To estimate this detector efficiency, a comparison is made between the total number of generated events, denoted as $N_\text{thrown}$, and the number of accepted events in a simulation of the detector, denoted as $N_\text{simul}$.
    This allows us to calculate an estimation of the detector efficiency, represented by $\tilde\xi_\text{tot}$, using
    \begin{equation*}
        \tilde\xi_\text{tot} = \frac{N_\text{simul}}{N_\text{thrown}}.
    \end{equation*}

    Naturally, the value of $\tilde\xi_\text{tot}$ is influenced by the accuracy and reliability of the event generator and simulation programs employed in the study.

% --+ Chosen bins +-------------------------------------------------------------
    The acceptance of the detector exhibits variations across the phase space of the kinematic variables.
    Therefore, in order to achieve accurate acceptance correction, the ratio $\tilde\xi_\text{tot}$ needs to be divided into bins in a five-dimensional space.
    These bins correspond to the five variables under investigation: $Q^2$, $\nu$, $z_h$, $p_T^2$, and $\phi_{PQ}$.
    To simplify the analysis process and facilitate interpretation of the results, it is advantageous to have bins of the same size for each variable.

    % TODO. I may need to update this list if I change these ranges in the future.
    \begin{itemize}
        \item
            $Q^2 = 4E_bE'\sin^2(\theta_C/2)$ is the 4-momentum transferred by the lepton probe in the lab frame, where $E_b$ is the beam energy, $E'$ is the scattered electron's energy, and $\theta_C$ is the polar angle of the scattered electron.
            The chosen bin edges are $1$, $2$, $3$, $4$, $5$, $6$, $7$, $8$, $9$, $10$, $11$ $\text{GeV}^2$.
        \item
            $\nu = E_v - E'$ is the energy transferred by the lepton probe in the lab frame.
            The chosen bin edges are $2$, $3$, $4$, $5$, $6$, $7$, $8$, $9$, and $10$ $\text{GeV}$.
        \item
            $z_h = E_h/\nu$ is the virtual photon energy fraction carried by the measured hadron, with $E_h$ being this hadron's energy.
            The chosen bin edges are $0$, $0.1$, $0.2$, $0.3$, $0.4$, $0.5$, $0.6$, $0.7$, $0.8$, $0.9$, and $1$.
        \item
            $p_T^2$ is the hadron's transverse momentum measured with respect to the virtual photon direction.
            The chosen bin edges are $0$, $0.2$, $0.4$, $0.6$, $0.8$, $1$, $1.2$, $1.4$, $1.6$, $1.8$, and $2$ $\text{GeV}^2$.
        \item
            $\phi_{PQ}$ is the angle between the leptonic plane -- the plane where the paths of the initial and scattered electrons lie -- and the hadronic plane, which contains the virtual photon and the measured hadron.
            The chosen bin edges are $-180$, $-140$, $-100$, $-60$, $-20$, $20$, $60$, $100$, $140$, and $180$ degrees.
    \end{itemize}

    To calculate the acceptance, 10 million events were initially generated in deep inelastic kinematics using LEPTO, a Monte Carlo generator specifically designed for deep inelastic lepton-nucleon scattering \cite{ingelman1997}.
    Subsequently, these events were simulated under the experimental conditions of the RG-F experiment in CLAS12 using \texttt{gemc}, which is the standard tool for CLAS12 event simulation \cite{ungaro2020gemc}.
    The simulation took into account a torus field polarity of $-1$ and a solenoid field polarity of $-0.745033$.

    Finally, the simulated events were reconstructed using \texttt{coatjava}, which is the standard tool for CLAS12 event offline reconstruction \cite{ziegler2020}.
    Further details regarding the offline reconstruction process can be found in Section \ref{sssec::offline_reconstruction}.
    The outcomes of the acceptance correction procedure are discussed in Section \ref{ssec::acceptance_correction_results}.
