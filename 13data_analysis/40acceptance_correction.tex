% !TEX root = ../main.tex
\subsection{Acceptance Correction}
\label{13.40::acceptance_correction}
% --+ What is acceptance +------------------------------------------------------
    When discussing radiation detection, it is customary to distinguish between two types of efficiency: absolute efficiency and intrinsic detection efficiency.
    The former is defined as the fraction of events emitted by the source that are actually detected by the detector, expressed as
    \begin{equation*}
        \xi_\text{tot} = \frac{\text{events registered}}{\text{events emitted by source}}.
    \end{equation*}

    This efficiency is influenced by the detector's geometry and the probability of an interaction occurring within the detector.
    The total efficiency is also referred to as the detector acceptance.

    The total efficiency can be further decomposed into two components: the intrinsic efficiency, $\xi_{\text{int}}$, and the geometric efficiency, $\xi_{\text{geom}}$.
    The total efficiency is then given by
    \begin{equation*}
        \xi_\text{tot} = \xi_\text{int} \cdot \xi_\text{geom}.
    \end{equation*}

    The intrinsic efficiency represents the fraction of events that actually reach and are detected by the detector
    \begin{equation*}
        \xi_\text{int} = \frac{\text{events registered}}{\text{events impinging on detector}}.
    \end{equation*}

    This probability is dependent on the interaction cross-sections of the incident radiation with the detector medium.
    The intrinsic efficiency thus varies with the type of radiation, its energy, and the detector material \cite{leo1987}.

% --+ Acceptance correction through generation + simulation +-------------------
    Acceptance correction involves compensating for the total efficiency of the detector.
    To estimate this detector efficiency, a comparison is made between the total number of generated events, denoted as $N_\text{thrown}$, and the number of accepted events in a simulation of the detector, denoted as $N_\text{simul}$.
    This allows us to calculate an estimation of the detector efficiency, represented by $\tilde\xi_\text{tot}$, using
    \begin{equation*}
        \tilde\xi_\text{tot} = \frac{N_\text{simul}}{N_\text{thrown}}.
    \end{equation*}

    Naturally, the value of $\tilde\xi_\text{tot}$ is influenced by the accuracy and reliability of the event generator and simulation programs employed in the study.

% --+ Chosen bins +-------------------------------------------------------------
    The acceptance of the detector exhibits variations across the phase space of the kinematic variables.
    Therefore, in order to achieve accurate acceptance correction, the ratio $\tilde\xi_\text{tot}$ needs to be divided into bins in a five-dimensional space.
    These bins correspond to the five variables under investigation: $Q^2$, $\nu$, $z_h$, $p_T^2$, and $\phi_{PQ}$.
    To simplify the analysis process and facilitate interpretation of the results, it is advantageous to have bins of the same size for each variable.

    \pagebreak

    \begin{itemize}
        \item
            For $Q^2$, given by Equation \eqref{eq::13.23::q2}, the lower edge of the binning is defined as $1 \text{ GeV}^2$, considering that a DIS cut of $Q^2 > 1$ is applied, as described in Section \ref{13.23::dis_cuts}.
            The upper edge is defined as $11 \text{ GeV}^2$, and the bin size is set to $1 \text{ GeV}^2$ to ensure sufficient statistics per bin.
            This results in 10 bins for $Q^2$.

        \item
            As shown in Figure \ref{fig::13.23::q2_vs_nu}, the minimum value of $\nu$ (described by Equation \eqref{eq::13.23::nu}) is $2 \text{ GeV}$ due to the $W^2 > 4$ cut, and its maximum value is $9 \text{ GeV}$ due to the $Y_b < 0.85$ cut.
            The bin size is set to $1 \text{ GeV}$, resulting in a total of 8 bins for $\nu$.

        \item
            The fraction $z_h$ of the virtual photon energy carried by the measured hadron is described in Section \ref{10.32::production_time} and given by Equation \eqref{eq::10.32::zh}.
            For this DIS study, we are interested in the region of this fraction ranging from $0$ to $1$.
            The bin size is defined as $0.1$, resulting in 10 bins for $z_h$.

        \item
            $p_T^2$ represents the hadron's transverse momentum measured with respect to the virtual photon direction.
            For this experiment, very few hadrons are observed with $p_T^2 > 2 \text{ GeV}^2$, thus the binning range is defined from $0$ to $2 \text{ GeV}^2$.
            The bin size is set to $0.2 \text{ GeV}^2$, resulting in 10 bins for $p_T^2$.

        \item
            $\phi_{PQ}$ is the angle between the leptonic plane -- the plane where the paths of the initial and scattered electrons lie -- and the hadronic plane, which contains the virtual photon and the measured hadron.
            It is important to be cautious about the binning scheme for this variable, as some geometric effects can arise from the 6-sector geometry of the CLAS12 Forward Detector (FD), as described in Section \ref{11.210::forward_detector}.
            Hence, 13 bins are defined from $-180 \degree$ to $180 \degree$, resulting in a bin size of approximately $27.7 \degree$.
    \end{itemize}

    To calculate the acceptance, 10 million events were initially generated in deep inelastic kinematics using LEPTO, a Monte Carlo generator specifically designed for deep inelastic lepton-nucleon scattering \cite{ingelman1997}.
    Subsequently, these events were simulated under the experimental conditions of the RG-F experiment in CLAS12 using \texttt{gemc}, which is the standard tool for CLAS12 event simulation \cite{ungaro2020gemc}.
    The simulation took into account a torus field polarity of $-1$ and a solenoid field polarity of $-0.745033$.

    Finally, the simulated events were reconstructed using \texttt{coatjava}, which is the standard tool for CLAS12 event offline reconstruction \cite{ziegler2020}.
    Further details regarding the offline reconstruction process can be found in Section \ref{11.230::offline_reconstruction}.
    The outcomes of the acceptance correction procedure are discussed in Section \ref{14.20::acceptance_correction_results}.

    % 10M events.
    % 104k bins.
    % = 96 events per 5D bin.
    % integrated, from 0.77M to 1.25M per bin.
