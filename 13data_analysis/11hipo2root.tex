% !TEX root = ../main.tex
\subsubsection{\texttt{hipo2root}}
\label{sssec::hipo2root}
    The CLAS12 reconstruction process utilises a custom data file format called High-Performance Input Output (HIPO) format, which was developed by Gagik Gavalian \cite{chekanov2021}.
    The CLAS collaboration provides a set of tools that enable working with HIPO files, allowing users to read, write, and generate plots from these files.
    However, users at RG-E are generally more familiar with traditional ROOT files. Therefore, a conversion tool was developed.

    The tool, called \texttt{hipo2root}, filters through a HIPO file's data and creates a ROOT file that includes pre-selected sections of storage known as banks.
    The selection of banks is based on data that is relevant to RG-E analysis, and users can easily add additional banks by modifying the program's source files.
    The output of the \texttt{hipo2root} executable is a ROOT file that contains the selected banks represented as trees, which are standard ROOT array-like variables.

    Its manual entry is:
    \begin{lstlisting}
Usage: hipo2root [-hfn:w:] infile
 * -h         : show this message and exit.
 * -f         : set this to true to process FMT::Tracks bank. If this is set
                and FMT::Tracks bank is not present in the HIPO file, the
                program will crash.
 * -n nevents : number of events.
 * -w workdir : location where output root files are to be stored. Default
                is root_io.
 * infile     : input HIPO file. Expected format is <text>run_no.hipo.

Convert a file from hipo to root format. This program only conserves the banks that are useful for RG-E analysis, as specified in the `lib/rge_hipo_bank.h` file. It's important for the input hipo file to specify the run number at the end of the filename (`<text>run_no.hipo`), so that `hipo2root` can get the beam energy from the run number.

Since simulation files don't have a run number, we use a convention for specifying the beam energy. For this files, the filename should be `<text>999XXX.hipo`, where `XXX` is the beam energy used in the simulation in [0.1*GeV].
    \end{lstlisting}
