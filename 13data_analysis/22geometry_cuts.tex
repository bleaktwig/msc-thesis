% !TEX root = ../main.tex
\subsubsection{Geometry Cuts}
\label{13.22::geometry_cuts}
    Three geometry cuts are derived to constrain the reconstructed particle's vertex position.
    The first two cuts ensure that the vertex is located along the beamline, while the third cut restricts it to the acceptance region of the FMT.

    The first cut guarantees that the vertex is close to the beamline and is defined as
    \begin{equation*}
        \sqrt{v_x^2 + v_y^2} < 4 \text{ cm},
    \end{equation*}
    where $v_x$ and $v_y$ represent the $x$ and $y$ coordinates of the vertex position, respectively.

    The second cut ensures that the vertex originates from the target and is given by
    \begin{equation*}
        -40 \text{ cm} < v_z < z_0 \text{ cm},
    \end{equation*}
    where $v_z$ corresponds to the $z$ coordinate of the vertex position, and $z_0$ represents the $z$ position of the first FMT layer.
    For the RG-F Spring 2020 run, $z_0 = 26.12$ cm.

    The third and final cut ensures that the vertex falls within the FMT acceptance region, as defined in Section \ref{12.42::geometry_effect}.
    It removes all particles whose $v_z$ and $\theta$ values lie outside the region bounded by the two lines defined by Equation \eqref{eq::12.42::fmt_geometry_cut}.

    Fiducial cuts were not applied in this analysis.
    The reasoning behind this decision is explained in Appendix \ref{20.02::fiducial_cuts}, along with a brief explanation of how the cuts would be applied.
