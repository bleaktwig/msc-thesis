% !TEX root = ../main.tex
\subsubsection{\texttt{draw\_plots}}
\label{sssec::draw_plots}
    The purpose of including the \texttt{draw\_plots} executable is to save the user from rewriting similar code repeatedly to obtain plots.
    Using the \texttt{draw\_plots} program is relatively straightforward: after running the program, the user is prompted with a series of questions to define various attributes related to the plots.
    This includes specifying any cuts and corrections to apply, setting up the binning, and selecting the variables to be plotted.
    Unless stated otherwise, the plots presented in section \ref{sec::results_and_conclusions} were generated using this executable.

    The executable's manual entry is:
    \begin{lstlisting}
Usage: draw_plots [-hp:cn:o:a:w:] infile
 * -h          : show this message and exit.
 * -p pid      : skip particle selection and draw plots for pid.
 * -c          : apply all cuts (general, geometry, and DIS) instead of
                 asking which ones to apply while running.
 * -n nentries : number of entries to process.
 * -o outfile  : output file name. Default is plots_<run_no>.root.
 * -a accfile  : apply acceptance correction using acc_filename.
 * -A          : get acceptance correction plots without applying
                 acceptance correction. Requires -a to be set.
 * -w workdir  : location where output root files are to be stored.
                 Default is root_io.
 * infile      : input file produced by make_ntuples.

Draw plots from a ROOT file built from `make_ntuples`. File should be named `<text>run_no.root`. This tool is built for those who don't enjoy using root too much, and should be able to get most basic plots needed in SIDIS analysis.
    \end{lstlisting}
