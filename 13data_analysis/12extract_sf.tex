% !TEX root = ../main.tex
\subsubsection{\texttt{extract\_sf}}
\label{sssec::extract_sf}
    Some analysis from CLAS12 is needed to obtain the sampling fraction of the detector's calorimeters.
    This is done by the \texttt{extract\_sf} executable, which uses the particles' momenta and deposited energy.
    The exact methodology and results obtained are discussed in section \ref{ssec::sampling_fraction}.

    The manual entry of the program is:
    \begin{lstlisting}
Usage: extract_sf [-hn:w:d:] infile
 * -h         : show this message and exit.
 * -n nevents : number of events
 * -w workdir : location where output root files are to be stored. Default
                is root_io.
 * -d datadir : location where sampling fraction files are stored. Default
                is data.
 * infile     : input ROOT file. Expected file format: <text>run_no.root.

Obtain the EC sampling fraction from an input file. An alternative to using this program is to fill the output file corresponding to the studied run (by default stored in the `data` directory) with the data obtained from [CCDB](https://clasweb.jlab.org/cgi-bin/ccdb/versions?table=/calibration/eb/electron_sf). The function used to fit the data is

[0]*Gaus(x,[1],[2]) + [3]*x*x + [4]*x + [5]

where *[0]* is the amplitude of the Gaussian, *[1]* and *[2]* its mean and sigma, and *[3]*, *[4]*, and *[5]* the *p0*, *p1*, and *p2* used to fit the background.

The output of the program is the `sf_params_<run_no>.txt`, which contains a table with the sampling fractions and their errors. The table is formatted like the one at CCDB, as in

         | sf0     sf1     sf2     sf3     sfs1    sfs2    sfs3    sfs4
---------+-----------------------------------------------------------------
sector 1 | %011.8f %011.8f %011.8f %011.8f %011.8f %011.8f %011.8f %011.8f
sector 2 | %011.8f %011.8f %011.8f %011.8f %011.8f %011.8f %011.8f %011.8f
sector 3 | %011.8f %011.8f %011.8f %011.8f %011.8f %011.8f %011.8f %011.8f
sector 4 | %011.8f %011.8f %011.8f %011.8f %011.8f %011.8f %011.8f %011.8f
sector 5 | %011.8f %011.8f %011.8f %011.8f %011.8f %011.8f %011.8f %011.8f
sector 6 | %011.8f %011.8f %011.8f %011.8f %011.8f %011.8f %011.8f %011.8f
    \end{lstlisting}
