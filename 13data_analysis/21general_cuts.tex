% !TEX root = ../main.tex
\subsubsection{General Cuts}
\label{13.21::general_cuts}
    For this analysis, only two cuts are considered as ``general''.
    The first cut involves filtering out particles with PID values of 0 or 45.
    In CLAS12 reconstruction, these specific PID values are assigned to particles whose identification could not be successfully determined.

    The second cut aims to exclude particles with imprecise tracking and is defined as follows:
    \begin{equation*}
        \frac{\chi^2}{\text{NDF}} < 15,
    \end{equation*}
    where $\chi^2$ represents the final result from the $\chi^2$-test used in the Kalman filter fit of the tracking algorithm, as described in Section \ref{11.230::offline_reconstruction}.
    The term NDF denotes the number of degrees of freedom associated with this same fit.

    By applying these two cuts, particles with undetermined or uncertain identification (PID values of 0 or 45) and those with poor tracking precision (exceeding the specified $\chi^2$/NDF threshold) are excluded from the analysis.
