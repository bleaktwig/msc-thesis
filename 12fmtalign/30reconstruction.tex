% !TEX root = ../main.tex
\subsection{FMT Reconstruction} \label{ssec::fmtreconstruction}
% --+ Why is it needed +--------------------------------------------------------
    As work in FMT alignment progressed, some changes were needed in FMT reconstruction.
    These changes mainly include the implementation of the shifts found in alignment to reconstruction.
    However, they also concern some fixes of issues found as alignment work progressed.

% --+ What was done +-----------------------------------------------------------
    First, the loading of shifts from the CCDB was included in the standard geometry class of the FMT reconstruction package.
    Then, standard methods to include the shifts in any frame of reference change were included.
    Finally, the code was studied in detail and the shits were added in all instance where they were required, since the package originally didn't consider their application.

    ``Crossmaking'' is the process of matching clusters in different layers to obtain an accurate 3D estimate of the position of a track \cite{ziegler2020}.
    This was originally included in FMT reconstruction to facilitate reconstruction for the six FMT layers.
    However, as was mentioned before, only three FMT layers were installed for the RG-F run, and future runs may also use three layers due to concerns with the Lorentz angle if six are used.
    To simplify reconstruction and make better use of this number of layers, crossmaking was removed from reconstruction.

    Outside of FMT reconstruction, some minor changes were also required in the DC reconstruction package, since some of its components depends on the FMT layers' positions.
    In addition, the swim package diagnostic was updated, since it failed to properly reconstruction the positions of tracks near the FMT layers.

% --+ Juicy link +--------------------------------------------------------------
    A detailed list of the updates applied can be found in the following pull request to the \texttt{clas12-offline-software} repository:

    \href{https://github.com/JeffersonLab/clas12-offline-software/pull/726}{\texttt{github.com/JeffersonLab/clas12-offline-software/pull/726}}.
