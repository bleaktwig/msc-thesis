% !TEX root = ../main.tex
\subsection*{Addendum 2: Reproducibility}
\addcontentsline{toc}{subsection}{Reproducibility}
\label{20.02::reproducibility}
    In order to ensure the reproducibility of the research presented in this thesis, we provide access to all datasets and code used in the development of our study.
    We believe that transparency and accessibility are crucial for scientific integrity and to facilitate further investigations by the research community.

    We encourage interested readers and fellow researchers to access and utilize these resources for the purpose of reproducibility and advancing scientific knowledge.
    Should there be any inquiries or issues regarding the datasets or code, please do not hesitate to contact the author at \href{mailto:bruno.benkel@gmail.com}{\texttt{bruno.benkel@gmail.com}} for further assistance.

    We believe that open access to data and code fosters collaboration, accelerates scientific progress, and ensures the robustness of research findings.
    By making these resources available, we aim to contribute to the collective effort of reproducible and transparent scientific research.

    % --+ Datasets +------------------------------------------------------------
    \paragraph{Datasets}
    Regrettably, there is no website or location to openly share datasets in the Universidad Técnica Federico Santa María (UTFSM) or the Centro Científico Tecnológico de Valparaíso (CCTVal).
    For readers with access to the JLab farm, all used datasets are available at:

    \begin{center}
        \texttt{/work/clas12/users/benkel/thesis-datasets}
    \end{center}

    For individuals who do not have access to the JLab farm, please feel free to contact the author, and we will explore alternative methods to share the relevant datasets.

    % --+ Code +----------------------------------------------------------------
    \paragraph{Code}
    The sources for the code used for data processing, analysis, and generating figures are shared on Table \ref{tab::20.02::code_locations}.
    By providing the code, we aim to enable researchers to replicate our findings, perform additional analyses, or build upon our work.

    \begin{table}[b!]
        \begin{center}
            \begin{tabularx}{0.90\textwidth}{ll}
                \toprule
                \textbf{Software}  & \textbf{Link} \\
                \midrule \midrule
                RG-E Slow Controls &
                    \href{https://github.com/bleaktwig/rge-epics-support}
                    {\texttt{github.com/bleaktwig/rge-epics-support}} \\
                \midrule
                CLAS12 Alignment   &
                    \href{https://github.com/JeffersonLab/clas12alignment}
                    {\texttt{github.com/JeffersonLab/clas12alignment}} \\
                \midrule
                thesis-simul       &
                    \href{https://github.com/bleaktwig/thesis-simul}
                    {\texttt{github.com/bleaktwig/thesis-simul}} \\
                thesis-data        &
                    \href{https://github.com/bleaktwig/thesis-data}
                    {\texttt{github.com/bleaktwig/thesis-data}} \\
                RG-E Analysis      &
                    \href{https://github.com/bleaktwig/clas12-rge-analysis}
                    {\texttt{github.com/bleaktwig/clas12-rge-analysis}} \\
                GEMC               &
                    \href{https://github.com/gemc/source}
                    {\texttt{github.com/gemc/source}} \\
                Coatjava*          &
                    \href{https://github.com/JeffersonLab/coatjava}
                    {\texttt{github.com/JeffersonLab/coatjava}} \\
                \bottomrule
            \end{tabularx}
        \end{center}

        \caption{Table with code locations.
        During the development of this thesis, \texttt{clas12-offline-software} was rebranded as Coatjava.}
        \label{tab::20.02::code_locations}
    \end{table}
