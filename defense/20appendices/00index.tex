% !TEX root = main.tex
\begin{frame}{}
    \centering \Huge{\ef{Appendices}}
\end{frame}

% --+ 20.01 CCDB +--------------------------------------------------------------
\begin{frame}{Calibration Constants Database}
    \label{20.01::ccdb}
    \begin{itemize}
        \item
        \vspace{12pt}
        In simple terms, the Calibration Constants Database (CCDB) is a central repository where the CLAS12 \ef{calibration constants} are stored.
        \item
            The stored data includes geometry constants, corrections for different imperfections, drifts in the electronics, etc.

        \vspace{12pt}
        \item
            A CCDB \ef{variation} (e.g. \eft{rgf\_spring2020}) is a ``version'' of the CCDB for a particular experiment season.

        \vspace{12pt}
        \item
            In addition, there are special variations for gemc simulations (e.g. \eft{rgf\_spring2020\_mc}), so as to not include calibration data from the real world into the simulation world.
    \end{itemize}

    \backref{11.42::alignment_effect}
\end{frame}

% --+ FMT DETAILS +-------------------------------------------------------------
% --+ FMT ALIGNMENT +-----------------------------------------------------------
% --+ SYSTEMATIC ERROR ESTIMATION +---------------------------------------------
% --+ REPRODUCIBILITY +---------------------------------------------------------
% --+ FIDUCIAL CUTS +-----------------------------------------------------------
% --+ FMT LAYER EFFICIENCY ERROR ESTIMATION +-----------------------------------
% --+ UNCORRECTED DIS PLOTS +---------------------------------------------------
