% !TEX root = main.tex

% --+ PRESENTATION DESIGN +-----------------------------------------------------
% Title page.
\title{Acceptance Study of the CLAS12 Detector Using the Forward Micromegas Tracker for Run Group E}
\author{Bruno Benkel}
\date{2023-09-14}

\defbeamertemplate*{title page}{customized}[1][]
{
    \vspace{21pt}

    \begin{center}
        \usebeamerfont{title}\usebeamercolor[color_ef]{title}\textbf{\inserttitle}\par
        \usebeamerfont{author}\usebeamercolor[color_fg]{author}\insertauthor\par
    \end{center}

    \vspace{48pt}

    \ef{Evaluating Committee}
    \begin{itemize}
        \item[]
            Hayk Hakobyan, Ph.D.

        \item[]
            Will Brooks, Ph.D.

        \item[]
            Raffaella De Vita, Ph.D.
    \end{itemize}

    \begin{flushright}
        \usebeamerfont{date}\insertdate\par
    \end{flushright}

    \vfill
}

% Footer - additional configuration found in main.tex, after the first pages.
\setbeamertemplate{navigation symbols}{}

% --+ PALETTE: EVERFOREST DARK HARD +-------------------------------------------
\definecolor{efd_bg}    {HTML}{272e33}
\definecolor{efd_fg}    {HTML}{d3c6aa}
\definecolor{efd_red}   {HTML}{e67e80}
\definecolor{efd_orange}{HTML}{e69875}
\definecolor{efd_yellow}{HTML}{dbbc7f}
\definecolor{efd_green} {HTML}{a7c080}
\definecolor{efd_dgreen}{HTML}{425047}
\definecolor{efd_aqua}  {HTML}{83c092}
\definecolor{efd_blue}  {HTML}{7fbbb3}
\definecolor{efd_purple}{HTML}{d699b6}

% --+ PRESENTATION COLORS +-----------------------------------------------------
% These four colors will be the palette of the presentation.
\colorlet{color_bg} {efd_bg}      % Background color of each slide.
\colorlet{color_fg} {efd_fg}      % Foreground color, mainly used by text.
\colorlet{color_ef} {efd_green}   % Emphasis color, uses for titles, etc.
\colorlet{color_box}{efd_dgreen} % Emphasis box color.

% Set the colors of the beamer presentation.
\setbeamercolor{normal text}       {bg=color_fg,fg=color_fg}
\setbeamercolor{background canvas} {bg=color_bg}
\setbeamercolor{structure}         {fg=color_ef} % itemize, enumerate, etc.
\setbeamercolor{item}              {bg=color_ef,fg=color_ef}
\setbeamercolor{palette primary}   {bg=color_ef,fg=color_bg}
\setbeamercolor{palette secondary} {bg=color_fg,fg=color_bg}
\setbeamercolor{palette tertiary}  {bg=color_fg,fg=color_bg}
\setbeamercolor{palette quaternary}{bg=color_fg,fg=color_bg}
\setbeamercolor{section in toc}    {fg=color_ef}

%%%%%%%%%%%%%%%%%%%%%%%%%%%%%%%%%%%%%%%%%%%%%%%%%%%%%%%%%%%%%%%%%%%%%%%%%%%%%%%%

% --+ CUSTOM COMMANDS +---------------------------------------------------------
% Equality sign with text above.
\newcommand{\labeq}{\stackrel{\mathclap{\normalfont\mbox{\tiny{lab}}}}{=}}
\newcommand{\lsim}{\lesssim} % \sim + >.
\newcommand{\gsim}{\gtrsim}  % \sim + <.

% Emphasis.
\newcommand{\ef}     [1]{\textbf{\textcolor{color_ef}{#1}}}   % Simple emphasis.
\newcommand{\eft}    [1]{\texttt{\ef{#1}}}                    % Emphasis + texttt.
\newcommand{\ered}   [1]{\textbf{\textcolor{efd_red}   {#1}}} % Red emphasis.
\newcommand{\eorange}[1]{\textbf{\textcolor{efd_orange}{#1}}} % Orange emphasis.
\newcommand{\eyellow}[1]{\textbf{\textcolor{efd_yellow}{#1}}} % Yellow emphasis.
\newcommand{\egreen} [1]{\textbf{\textcolor{efd_green} {#1}}} % Green emphasis.
\newcommand{\eaqua}  [1]{\textbf{\textcolor{efd_aqua}  {#1}}} % Aqua emphasis.
\newcommand{\eblue}  [1]{\textbf{\textcolor{efd_blue}  {#1}}} % Blue emphasis.
\newcommand{\epurple}[1]{\textbf{\textcolor{efd_purple}{#1}}} % Purple emphasis.

% Ref from presentation to backup slides.
\newcommand{\appref}[1]{\textsuperscript{\textcolor{efd_purple}{[\ref{#1}]}}}

% Ref from backup slides to presentation.
\newcommand{\backref}[1]{
    \vskip0pt plus 1 filll
    \begin{flushright}
        \tiny{\textit{Ref'd by \textcolor{efd_green}{\ref{#1}}.}}
    \end{flushright}
    \smallskip
}

% Fancy empheq box.
\newlength\mytemplen
\newsavebox\mytempbox

\makeatletter
\newcommand\eqbox{
    \@ifnextchar[
        {\@eqbox}
        {\@eqbox[0pt]}
}

\def\@eqbox[#1]{
    \@ifnextchar[
        {\@@eqbox[#1]}
        {\@@eqbox[#1][0pt]}
}

\def\@@eqbox[#1][#2]#3{
    \sbox\mytempbox{#3}
    \mytemplen\ht\mytempbox
    \advance\mytemplen #1\relax
    \ht\mytempbox\mytemplen
    \mytemplen\dp\mytempbox
    \advance\mytemplen #2\relax
    \dp\mytempbox\mytemplen
    \colorbox{color_box}{\hspace{1em}\usebox{\mytempbox}\hspace{1em}}
}

\makeatother

% Custom overline (from tex.stackexchange.com/questions/22100/the-bar-and-overline-commands).
\makeatletter
\newsavebox\myboxA
\newsavebox\myboxB
\newlength\mylenA

\newcommand*\xoverline[2][0.75]{%
    \sbox{\myboxA}{$\m@th#2$}%
    \setbox\myboxB\null% Phantom box
    \ht\myboxB=\ht\myboxA%
    \dp\myboxB=\dp\myboxA%
    \wd\myboxB=#1\wd\myboxA% Scale phantom
    \sbox\myboxB{$\m@th\overline{\copy\myboxB}$}% Overlined phantom
    \setlength\mylenA{\the\wd\myboxA}% calc width diff
    \addtolength\mylenA{-\the\wd\myboxB}%
    \ifdim\wd\myboxB<\wd\myboxA%
       \rlap{\hskip 0.5\mylenA\usebox\myboxB}{\usebox\myboxA}%
    \else
        \hskip -0.5\mylenA\rlap{\usebox\myboxA}{\hskip 0.5\mylenA\usebox\myboxB}%
    \fi
}

% --+ MISCELLANEOUS +-----------------------------------------------------------
\setlength{\fboxsep}{0pt} % Tight fboxes.
