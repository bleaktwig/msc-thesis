% !TEX root = ../main.tex
% --+ 20.03 FMT EFFICIENCY ERROR ESTIMATION +-----------------------------------
\begin{frame}{FMT Efficiency Error Estimation}
    \label{20.03::fmt_efficiency_error_estimation}

    \vspace{18pt}

    We define \ef{$P(L_n)$} as the probability of layer \ef{$n$} detecting a particle.
    Assuming all layers have the same efficiency \ef{$E_1$},
    \begin{equation*}
        P(L_1) = P(L_2) = P(L_3) = E_1,
    \end{equation*}
    the efficiency for 3-layer tracks \ef{$E_3$} can be obtained using \ef{$P(L_n)$} as follows
    \begin{align}
        E_3 &= P(L_1)P(L_2)P(L_3)
        \nonumber \\
        E_3 &= E_{1(3)}^3,
        \label{eq::14.14::efficiency3}
    \end{align}
    where \ef{$E_{1(3)}$} is the 1-layer efficiency estimated from 3-layer tracks.

    \vspace{18pt}

    From \eqref{eq::14.14::efficiency3}, we can estimate \ef{$E_{1(3)}$} as
    \begin{equation}
        E_{1(3)} = \sqrt[3]{E_3}.
        \label{eq::14.14::efficiency1(3)}
    \end{equation}

    \backref{11.45::reconstruction_effect}
\end{frame}

\begin{frame}{FMT Efficiency Error Estimation}
    \vspace{18pt}

    For 2-layer tracks, the efficiency \ef{$E_2$} can be obtained as
    \begin{align}
        E_2 &= P(L_1)P(L_2)\left(1 - P(L_3)\right)                \nonumber \\
             &\hspace{24pt} + P(L_2)P(L_3)\left(1 - P(L_1)\right) \nonumber \\
             &\hspace{24pt} + P(L_3)P(L_1)\left(1 - P(L_2)\right) \nonumber \\
             &\hspace{24pt} + P(L_1)P(L_2)P(L_3)                  \nonumber \\
        E_2 &= 3E_{1(2)}^2\left(1 - E_{1(2)}\right) + E_{1(3)}^3
            \nonumber \\
        E_2 &= 3E_{1(2)}^2 \cdot \left( 1 - E_{1(2)} \right) + E_3,
        \label{eq::14.14::efficiency2}
    \end{align}
    where \ef{$E_{1(2)}$} is the 1-layer efficiency estimated from 2-layer tracks.

    \vspace{18pt}

    \ef{$E_{1(2)}$} cannot be obtained explicitly from \eqref{eq::14.14::efficiency2}, but we can estimate it numerically.

    \backref{11.45::reconstruction_effect}
\end{frame}

\begin{frame}{FMT Efficiency Error Estimation}
    \vspace{18pt}

    Using Equations \eqref{eq::14.14::efficiency2} and \eqref{eq::14.14::efficiency1(3)}, we can estimate the weighted average efficiency \ef{$\xoverline{E_1}$} as
    \begin{equation*}
        \xoverline{E_1} = \frac{4E_{1(2)} + E_{1(3)}}{5},
    \end{equation*}
    where the weights are assigned based on the number of ways 2 and 3-layer tracks can be obtained, 4 and 1, respectively.

    \vspace{18pt}

    From \ef{$\xoverline{E_1}$}, we can estimate the errors on \ef{$E_{1(2)}$} and \ef{$E_{1(3)}$} as
    \begin{align*}
        \delta(E_{1(2)}) = |\xoverline{E_1} - E_{1(2)}|, \\
        \delta(E_{1(3)}) = |\xoverline{E_1} - E_{1(3)}|.
    \end{align*}

    \backref{11.45::reconstruction_effect}
\end{frame}

\begin{frame}{FMT Efficiency Error Estimation}
    To propagate these errors to the efficiencies \ef{$E_2$} and \ef{$E_3$}, we use the variance formula
    \begin{equation*}
        \delta\left(f(x)\right) = \frac{\partial f(x)}{\partial x} \cdot \delta(x),
    \end{equation*}
    where \ef{$\delta(E_2)$}, obtained from Equation \eqref{eq::14.14::efficiency2}, is
    \begin{align*}
        \delta(E_2) &= \frac{\partial}{\partial E_{1(2)}} \left( 3E_{1(2)}^2 - 3E_{1(2)}^3 + E_{1(3)}^3 \right)
            \cdot \delta \left( E_{1(2)} \right) \\
        \delta(E_2) &= \left( 6E_{1(2)} - 9E_{1(2)}^2 \right) \cdot \delta \left( E_{1(2)} \right)
    \end{align*}
    and \ef{$\delta(E_3)$}, obtained from Equation \eqref{eq::14.14::efficiency3}, is
    \begin{align*}
        \delta(E_3) &= \frac{\partial}{\partial E_{1(3)}} \left( E_{1(3)}^3 \right) \cdot \delta \left( E_{1(3)} \right) \\
        \delta(E_3) &= 3E_{1(3)}^2 \cdot \delta \left( E_{1(3)} \right).
    \end{align*}

    \backref{11.45::reconstruction_effect}
\end{frame}
