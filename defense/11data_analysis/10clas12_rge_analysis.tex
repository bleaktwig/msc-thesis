% !TEX root = ../main.tex
% --+ 11.11 CLAS12 RG-E ANALYSIS +----------------------------------------------
\begin{frame}{\texttt{clas12-rge-analysis}}
    \label{11.11::clas12-rge-analysis}

    To perform the analysis, we developed an extensive C/C++ library that runs over ROOT.
    The library includes five executables:

    \vspace{12pt}

    \begin{itemize}
        \item
            \eft{hipo2root} converts a HIPO file into ROOT trees, filtering out irrelevant data.

        \vspace{3pt}
        \item
            \eft{extract\_sf} obtains the $e^-$ sampling fraction for its posterior use.

        \vspace{3pt}
        \item
            \eft{acc\_corr} obtains acceptance correction from simulation in 5-dimensional bins.

        \vspace{3pt}
        \item
            \eft{make\_ntuples} generates tuples with particle data for DIS analysis.

        \vspace{3pt}
        \item
            \eft{draw\_plots} quickly draws plots from the tuples based on user input.
    \end{itemize}

    \vspace{24pt}

    The program and its source code are shared under the GNU LGPLv3 license\appref{12.13::reproducibility}.
\end{frame}

% --+ 11.12 PARTICLE IDENTIFICATION TRUTH +-------------------------------------
\begin{frame}{Particle Identification Truth}
    \label{11.12::particle_identification_truth}

    \begin{itemize}
        \item
            The \eft{make\_ntuples} program assigns a Particle ID (PID) to each particle in the same manner as in CLAS12 reconstruction.

        % \item
        %     Misidentification in this assignment is an important source of systematic errors.

        \item
            To quantify misidentification, we can run a simulation to compare the program's identification with the original PID in the simulation.
    \end{itemize}

    \begin{center}
        \begin{tabularx}{230pt}{Xllllll}
            \toprule
                          & \ef{$e$} & \ef{$\pi$} & \ef{$K$} & \ef{$p$} & \ef{$n$} & \ef{$\gamma$} \\
            \midrule
            % Recon.
            \ef{$e$}      &     0.98 &            &          &          &          &               \\
            \ef{$\pi$}    &          &       0.93 &     0.10 &     0.00 &          &               \\
            \ef{$K$}      &          &       0.03 &     0.80 &     0.00 &          &               \\
            \ef{$p$}      &          &       0.03 &     0.02 &     0.98 &          &               \\
            \ef{$n$}      &          &            &          &          &     0.66 &          0.01 \\
            \ef{$\gamma$} &          &            &          &          &     0.14 &          0.95 \\
            \bottomrule

            % clas12-rge-analysis
            % $e$           &     1.00 &            &          &          &          &               \\
            % $\pi$         &          &      1.00  &     0.09 &     0.02 &          &               \\
            % $K$           &          &            &     0.91 &          &          &               \\
            % $p$           &          &            &          &     0.98 &          &               \\
            % $n$           &          &            &          &          &     1.00 &               \\
            % $\gamma$      &          &            &          &          &          &          1.00 \\

            % Compounded.
            % $e$           &     0.98 &            &          &          &          &               \\
            % $\pi$         &          &      0.93  &     0.13 &     0.02 &          &               \\
            % $K$           &          &      0.03  &     0.86 &     0.00 &          &               \\
            % $p$           &          &      0.03  &     0.02 &     0.98 &          &               \\
            % $n$           &          &            &          &          &     0.66 &          0.01 \\
            % $\gamma$      &          &            &          &          &     0.14 &          0.95 \\
        \end{tabularx}
    \end{center}
    \scriptsize{\textit{
        Particle identification truth matrix.
        Rows are the PID truth, Columns are the PID as identified by \eft{make\_ntuples}.
        Diagonal elements are correctly identified, off-diagonal elements are misidentified.
    }}
\end{frame}
