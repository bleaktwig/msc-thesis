% !TEX root = ../main.tex
% --+ 11.41 SUMMARY +-----------------------------------------------------------
\begin{frame}{FMT Efficiency}
    \label{11.41::summary}
    \begin{itemize}
        \item
            Initially, my plan was to work with run \ef{12933} (10.4 GeV beam, 250 nA), but analysis shows a very poor FMT efficiency.

        \item
            This issue comes from three sources: \ef{alignment}, \ef{reconstruction}, and \ef{geometry}.
    \end{itemize}

    \vspace{-12pt}
    \begin{columns}[onlytextwidth,T]

    \begin{column}{.05\linewidth}\end{column} % Centering column.

    \begin{column}{.49\linewidth}
        \begin{center}
            \begin{figure}[t]
                \centering{
                    \fbox{\includegraphics[width=\textwidth]{41vz_011983.png}}
                }
                \textit{Run \ef{11983}, reconstructed in 2020.}
            \end{figure}
        \end{center}
    \end{column}

    \begin{column}{.29\linewidth}
        \begin{center}
            \begin{figure}[t]
                \centering{
                    \fbox{\includegraphics[width=\textwidth]{41vz_012933.png}}
                }
                \textit{Run \ef{12993}, 2023.}
            \end{figure}
        \end{center}
    \end{column}

    \begin{column}{.05\linewidth}\end{column} % Centering column.

    \end{columns}
    \begin{center}
        \textit{\ef{$v_z$} for \textbf{\textcolor[HTML]{c7eca6}{DC (green)}} and \textbf{\textcolor[HTML]{8dcfbf}{FMT (cyan)}} tracks.}
    \end{center}
\end{frame}

% --+ 11.42 ALIGNMENT EFFECT +--------------------------------------------------
\begin{frame}{FMT Efficiency: Alignment Effect}
    \label{11.42::alignment_effect}
    \begin{itemize}
        \item
            Alignment problem comes from an issue with the \ef{Summer 2020} alignment table in the Calibration Constants Database\appref{20.01::ccdb}.

        \item
            To work around this, we switch to \ef{Spring 2020} data.
    \end{itemize}

    \vspace{-12pt}
    \begin{columns}[onlytextwidth,T]

    \begin{column}{.05\linewidth}\end{column} % Centering column.

    \begin{column}{.35\linewidth}
        \begin{center}
            \begin{figure}[t]
                \centering{
                    \fbox{\includegraphics[width=\textwidth]{41vz_012933.png}}
                }
                \textit{Summer run \ef{12933}.}
            \end{figure}
        \end{center}
    \end{column}

    \begin{column}{.35\linewidth}
        \begin{center}
            \begin{figure}[t]
                \centering{
                    \fbox{\includegraphics[width=\textwidth]{42vz_012016.png}}
                }
                \textit{Spring run \ef{12016}.}
            \end{figure}
        \end{center}
    \end{column}

    \begin{column}{.05\linewidth}\end{column} % Centering column.

    \end{columns}
    \begin{center}
        \textit{\ef{$v_z$} for \textbf{\textcolor[HTML]{c7eca6}{DC (green)}} and \textbf{\textcolor[HTML]{8dcfbf}{FMT (cyan)}} tracks.}
    \end{center}
\end{frame}

% --+ 11.43 GEOMETRY EFFECT +---------------------------------------------------
\begin{frame}{FMT Efficiency: Geometry Effect}
    \label{11.43::geometry_effect}
    \begin{itemize}
        \item
            Due to its position, FMT has a non-trivial efficiency curve along the \ef{$z$ axis} and \ef{$\theta$ angle}.

        \vspace{6pt}
        \item
            If we want to understand the behavior of DIS variables, we need to study this effect and disentangle it from acceptance.
    \end{itemize}

    \vspace{12pt}
    We can define this \ef{efficiency region} by projecting straight lines between the $z$ axis and the detector\appref{20.02::fmt_acceptance_curve}, such that
    \begin{empheq}[box={\eqbox[5pt][5pt]}]{equation*}
        \theta_\text{min}(z) = 57.29^\circ \cdot \text{atan}\left( \frac{R_\text{min}}{z_0 - z} \right),
        \hspace{10pt}
        \theta_\text{max}(z) = 57.29^\circ \cdot \text{atan}\left( \frac{R_\text{max}}{z_0 - z} \right),
    \end{empheq}
    where \ef{$R_\text{min}$} and \ef{$R_\text{max}$} are the radii of FMT, and \ef{$z_0$} is the first layer's $z$ position.

    \vspace{6pt}
    We then apply a cut on DC and FMT tracks based on this region.
\end{frame}

