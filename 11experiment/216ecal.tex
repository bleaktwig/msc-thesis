% !TEX root = ../main.tex
\paragraph{Electromagnetic Calorimeters (ECAL)}
    The CLAS12 package incorporates the existing Electromagnetic Calorimeter (EC) from the CLAS detector and adds a new PCAL upstream of it.
    Together, they form the ECAL, which is primarily used for the identification and kinematical reconstruction of electrons, photons, and neutrons.

    The ECAL consists of six modules, with the PCAL and EC divided into two parts each along the direction from the target.
    These parts are known as EC-inner (ECIN) and EC-outer (ECOU) and are read out separately.
    They provide longitudinal sampling of electromagnetic showers and also help improve particle identification through hadronic interactions.

    Each module has a triangular shape and is composed of 54 layers of scintillators.
    The scintillators are 1 cm thick and segmented into strips that are 4.5 cm wide for PCAL and 10 cm wide for EC, fitted between 2.2-mm-thick lead sheets.
    The total thickness of the calorimeters corresponds to approximately 20.5 radiation lengths.
    The scintillator layers are grouped into three readout views, with 5/5/8 layers per view for PCAL/ECIN/ECOU, respectively.
    This arrangement allows for spatial resolutions of less than 2 cm for energy clusters.

    To transmit the light signals from the scintillators, flexible optical fibres are used to route the light to the corresponding PMTs \cite{asryan2020}.
