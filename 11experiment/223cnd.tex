% !TEX root = ../main.tex
\paragraph{Central Neutron Detector (CND)}
    The Central Neutron Detector (CND) is a component of the CLAS12 Central Detector (CD) positioned radially outward of the CTOF system.
    Its primary function is to detect neutrons in the momentum range of $0.2$ to $1.0 \text{GeV}$ by measuring their TOF from the target and the energy deposition in scintillator layers.

    The CND is composed of three layers of scintillator paddles, with each layer containing 48 paddles.
    The paddles are coupled in pairs at the downstream end using semi-circular light guides.
    The signal generated in the scintillator paddles is read out at the upstream end by PMTs that are positioned outside the high magnetic field region of the solenoid magnet.

    To transmit the scintillation light, the scintillators are connected to 1-meter-long bent light guides, which ensure efficient light propagation to the PMTs for signal detection and readout.

    The combination of TOF measurements and energy deposition in the scintillator layers enables the CND to identify and detect neutrons within the specified momentum range in the CLAS12 CD \cite{chatagnon2020}.
