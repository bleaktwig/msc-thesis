% !TEX root = ../main.tex
\paragraph{Central Time-of-Flight}
    The CTOF system is utilized for the purpose of charged particle identification by measuring their time-of-flight (TOF) in the momentum range of approximately $0.3$ to $1.25 ~\text{GeV}$.
    It consists of 48 plastic scintillators with double-sided photomultiplier tube (PMT) readout.
    The PMTs are connected to the scintillators via 1.0-meter-long upstream and 1.6-meter-long downstream focusing light guides.
    These counters are arranged in a hermetic barrel configuration surrounding the target and the CVT, and they are aligned with the beam axis inside the 5 T solenoid magnet.

    To ensure accurate measurements, the PMTs are positioned within a region of 0.1 T fringe field of the solenoid magnet and are enclosed within a triple-layer dynamic magnetic shield.
    This shield minimizes the internal magnetic field near the PMT photocathode, achieving a field strength of less than 0.2 G.
    The CTOF system is designed to provide a time resolution of 80 ps, enabling precise charged particle identification in the CLAS12 CD \cite{carman2020ctof}.

    For a visual representation of the CTOF system, you can refer to figure \ref{fig::ctof}.
