% !TEX root = ../main.tex
\paragraph{Alarm System}
    To ensure the reliable operation of the CLAS12 detector, all EPICS-controlled subsystems within it are equipped with PVs that define alarm conditions.
    These alarms, along with their severity, associated subsystems, and pre-defined instructions on how to respond to them, are displayed in a centralised alarm system.

    For each experiment conducted in Hall B involving a non-trivial target system, a specific set of alarms is required.
    In the case of the RG-E target, the list of implemented alarms and their corresponding PVs is provided in Table \ref{tab::11.313::alarms_pv}.

    \begin{table}[b!]
        \begin{center}
            \begin{tabularx}{360pt}{llX}
                \toprule
                \textbf{Name}          & \textbf{Trigger}                     & \textbf{Severity} \\
                \midrule \midrule
                Band is not moving     & \texttt{DMC01:A\_tgttype}       =  0 & Major             \\
                Band is moving         & \texttt{DMC01:A\_tgttype}       =  9 & Minor             \\
                Band beyond low limit  & \texttt{DMC01:A\_tgttype}       = 10 & Major             \\
                Band beyond high limit & \texttt{DMC01:A\_tgttype}       = 11 & Major             \\
                Controller comm. error & \texttt{DMC01:COMMERR\_STATUS}  =  1 & Major             \\
                IOC comm. error        & \texttt{IOC01:SR\_i\_am\_alive} =  0 & Major             \\
                \bottomrule
            \end{tabularx}
        \end{center}

        \caption[RG-E slow controls system alarm list.]
        {Alarm names, triggers, and severities for the RG-E target slow control system.}
        \label{tab::11.313::alarms_pv}
    \end{table}
