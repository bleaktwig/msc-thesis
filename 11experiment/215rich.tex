% !TEX root = ../main.tex
\paragraph{Ring Imaging Cherenkov Detector (RICH)}
\label{par::rich}
    To improve particle identification in the momentum range of $3 - 8 ~\text{GeV}$, a Ring Imaging Cherenkov (RICH) detector was incorporated into one of the CLAS12 sectors, replacing the corresponding LTCC sector.
    The RICH detector enhances CLAS12's capabilities in separating kaons from pions.

    The RICH detector consists of aerogel radiators, visible light photon detectors, and a focusing mirror system.
    The focusing mirror system is designed to reduce the instrumented detection area to $1 ~\text{m}^2$.
    Multi-anode Photomultiplier Tubes (MaPMTs) are used as the photon detectors, providing the necessary spatial resolution and matching the aerogel Cherenkov light spectrum in the visible and near-UV region.

    For forward scattered particles with momenta between $3$ and $8 ~\text{GeV}$ and angles up to $13\degree$, a proximity imaging method is employed.
    This method utilises a thin ($2 ~\text{cm}$) aerogel radiator for direct Cherenkov light detection.

    For particles with larger incident angles between $13\degree$ and $25\degree$ and momenta ranging from $3$ to $6 ~\text{GeV}$, a different configuration is used.
    The Cherenkov light is produced by a thicker aerogel layer of $6 ~\text{cm}$ and focused by a spherical mirror.
    The light then undergoes two additional passes through thin radiator material and is reflected by planar mirrors before detection \cite{contalbrigo2020}.
    This setup enables efficient particle identification in this momentum and angular range.
