% !TEX root = ../main.tex
\paragraph{Ring Imaging Cherenkov Detector (RICH)}
    For momenta greater than 3 GeV, the TOF resolution of FTOF is not sufficient to separate kaons from pions.
    For that puspose, an additional RICH detector was built and incorporated into one of the CLAS12 sectors to replace the corresponding LTCC sector.
    The RICH detector is designed to improve CLAS12 particle identification in the momentum range $3 - 8 ~\text{GeV}$.
    The detector incorporates aerogel radiators, visible light photon detectors, and a focusing mirror system that is used to reduce the detection area instrumented by photon detectors to $1 ~\text{m}^2$.

    Multi-anode PhotoMultiplier Tubes (MaPMTs) provide the required spatial resolution and match the aerogel Cherenkov light spectrum in the visible and near-UV region.
    For forward scattered particles up to $13\degree$ with momenta $3 - 8 ~\text{GeV}$, a proximity imaging method with thin ($2 ~\text{cm}$) aerogel and direct Cherenkov light detection is used.
    For larger incident particle angles between $13\degree$ and $25\degree$ and momenta of $3 - 6 ~\text{GeV}$, the Cherenkov light is produced by a thicker aerogel layer of $6 ~\text{cm}$, focused by a spherical mirror, and undergoes two further passes through the thin radiator material and a reflection from planar mirrors before detection \cite{contalbrigo2020}.
