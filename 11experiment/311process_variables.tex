% !TEX root = ../main.tex
% --+ ai +----------------------------------------------------------------------
\paragraph{Analog Input (\texttt{ai})}
\label{par::analog_input}
    The normal use for this record type is to obtain an analog value from hardware and then convert it to engineering units \cite{stanley1998}.
    The record can also be used to write constants to be read from the database, such that they can be changed in runtime.

    \subparagraph{\texttt{DMC01:A\_curr\_pos}.}
        Current position of the band.
        Displayed at the GUI and used for internal calculations.

    \subparagraph{\texttt{DMC01:A\_home}.}
        Position of the home.
        Displayed at the GUI and used for calculations.

    \subparagraph{\texttt{DMC01:A\_pos\#}.}
        \# is a number from 1 to 7.
        Positions of each of the seven targets.
        Displayed at the GUI and used for calculations.

    \subparagraph{\texttt{DMC01:A\_lowlimit}.}
        Position of the low limit.
        If \texttt{DMC01:A\_curr\_pos} is lesser than this value, a major alarm is fired.

    \subparagraph{\texttt{DMC01:A\_highlimit}.}
        Position of the high limit.
        If \texttt{DMC01:A\_curr\_pos} is greater than this value, a major alarm is fired.

    \subparagraph{\texttt{DMC01:A\_tolerance}.}
        Equivalence tolerance for the position of each target and the position of the band.
        It defines a valid range around the target position.

    \subparagraph{\texttt{DMC01:COMMERR\_STATUS}.}
        Variable that is true when there's a communication error with the controller and false otherwise.
        Used for triggering a communication alarm.

    \subparagraph{\texttt{IOC01:SR\_i\_am\_alive}.}
        Variable that is true when the IOC is up and running, false otherwise.
        Used for triggering a communication alarm.

% --+ ao +----------------------------------------------------------------------
\paragraph{Analog Output (\texttt{ao})}
\label{par::analog_output}
    The normal use for this record type is to output values to digital-analog converters.
    The desired output can be controlled by either an operator or a state program, or it can be fetched from another record \cite{stanley1998}.

    \subparagraph{\texttt{DMC01:A\_go\_home}.}
        Command to move the band to the home position, as defined in \texttt{DMC01:A\_home}.

    \subparagraph{\texttt{DMC01:A\_go\_pos\#}.}
        Command to move the band to the position of target \#, as defined in \texttt{DMC01:A\_pos\#}.

% --+ calc +--------------------------------------------------------------------
\paragraph{Calculation (\texttt{calc})}
\label{par::calculation}
    The calculation or ``Calc'' record is used to perform algebraic, relational, and logical operations on values retrieved from other records.
    The result of its operations can then be accessed by another record so that it can be used \cite{stanley1998}.
    In the context of the RG-E target, each calculation returns a number from 0 to 11.
    This number represent the state the target is in, and is later used by a Select PV.

    \subparagraph{\texttt{DMC01:A\_at\_pos\#}.}
        Calculation that checks if the band position is equal to that of target \# in \texttt{DMC01:A\_pos\#}, within the tolerance margin \texttt{DMC01:A\_tolerance}.
        If it is, it returns \#.
        Otherwise, it returns 0.

    \subparagraph{\texttt{DMC01:A\_at\_home}.}
        Calculation that checks if the band position is equal to the home position in \texttt{DMC01:A\_home}, within the tolerance margin defined by the tolerance.
        If it is, it returns 8.
        Otherwise, it returns 0.

    \subparagraph{\texttt{DMC01:A\_moving}.}
        Calculation that checks if the target is moving by checking the motor PV \texttt{DMC01:A.MOVN}.
        If it is, it returns 9.
        Otherwise, it returns 0.

    \subparagraph{\texttt{DMC01:A\_at\_lowlimit}.}
        Calculation that checks if the band position is lesser than the low limit \texttt{DMC01:A\_lowlimit}.
        If it is, it returns 10.
        Otherwise, it returns 0.

    \subparagraph{\texttt{DMC01:A\_at\_highlimit}.}
        Calculation that checks if the band position is greater than the high limit \texttt{DMC01:A\_highlimit}.
        If it is, it returns 11.
        Otherwise, it returns 0.

% --+ sel +---------------------------------------------------------------------
\paragraph{Select (\texttt{sel})}
\label{par::select}
    The select record computes a value based on input obtained from up to 12 locations.
    By default, it is equal to the highest value among its input PVs \cite{stanley1998}.

    \subparagraph{\texttt{DMC01:A\_sel\_tgttype}.}
        This record returns the highest value between the previously defined calculations.
        Thus, it associates the values returned to a state of the target.
        By convention, this PV assumes that \emph{no more than one \texttt{calc} is greater than 0}.
        This assumptions holds as long as \texttt{DMC01:A\_tolerance} is not set higher than half the distance between targets and between the targets and the lower and higher limits.

% --+ mbbi +--------------------------------------------------------------------
\paragraph{Multi-Bit Binary Input (\texttt{mbbi})}
\label{par::multibit_binary_input}
    The normal use for the multi-bit binary input record is to read multiple bit inputs from hardware.
    The binary value represents a state from a range of up to 16 states.
    The multi-bit input record interfaces with devices that use more than one bit \cite{stanley1998}.

    \subparagraph{\texttt{DMC01:A\_tgttype}.}
        This \texttt{mbbi} encodes the output of \texttt{DMC01:A\_sel\_tgttype} to a string and alarm level.
        The encoding is specified in table \ref{tab::target_types_pv}.

% --+ Everything table +--------------------------------------------------------
\begin{table}[b!]
    \caption{Names and alarm levels for the different values of the PV \texttt{DMC01:A\_tgttype}.}

    \begin{center}
        \begin{tabularx}{220pt}{lll}
            \hline
            \textbf{Value} & \textbf{Name} & \textbf{Alarm Severity} \\
            \hline
             0             & Not Moving    & Major                   \\
             1             & Target 1      & No alarm                \\
             2             & Target 2      & No alarm                \\
             3             & Target 3      & No alarm                \\
             4             & Target 4      & No alarm                \\
             5             & Target 5      & No alarm                \\
             6             & Target 6      & No alarm                \\
             7             & Target 7      & No alarm                \\
             8             & Home          & No alarm                \\
             9             & Moving        & Minor                   \\
            10             & Low Limit     & Major                   \\
            11             & High Limit    & Major                   \\
            \hline
        \end{tabularx}
    \end{center}
    \label{tab::target_types_pv}
\end{table}
