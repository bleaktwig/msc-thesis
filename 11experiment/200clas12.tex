% !TEX root = ../main.tex
\subsection{CLAS12}
\label{11.200::clas12}
    \begin{figure}[b!]
        \frame{\includegraphics[width=\textwidth]{200clas12_diagram.png}}
        \caption[CLAS12.]
        {The CLAS12 detector in the Hall B beamline.
        The electron beam enters from the right and impinges on the target located in the center of the solenoid magnet shown at the right (upstream) end of CLAS12.
        The FD consists of the High Threshold Cherenkov Counter (HTCC) (yellow), the torus magnet (grey), the Drift Chambers (DC) tracking system (light blue), and another set of Cherenkov counters (hidden), time-of-flight scintillation counters (brown), and Electromagnetic Calorimeters (ECs) (red).
        Between the HTCC and the torus, the Forward Tagger (FT) is installed.
        The Central Detector (CD) consists of the Silicon Vertex Tracker (SVT) (hidden), which is surrounded by a Barrel Micromesh Tracker (BMT) (hidden), the Central Time-of-Flight (FTOF) system, and the Central Neutron Detector (CND) (blue).
        At the upstream end, a Back Angle Neutron Detector (BAND) (red) is installed.}
        \floatfoot{Source: \href{https://jlab.org/physics/hall-b/clas12}{CLAS12 wiki}.}
        \label{fig::11.200::clas12_diagram}
    \end{figure}

    The main detector in Hall B is CLAS12, used to study electro-induced nuclear and hadronic reactions \cite{burkert2020}.
    The spectrometer provides efficient detection of charged and neutral particles over a large fraction of the full solid angle.

    CLAS12 is based on two superconducting magnets: a solenoid magnet and a 5 T torus magnet.
    The detector is divided into two parts: the Forward Detector (FD) and the Central Detector (CD).
    The FD, aided by the torus magnet, covers the forward polar range from $5\degree$ up to $35\degree$, while the CD, aided by the solenoid magnet, covers the polar angles from $35\degree$ to $125\degree$.
    Both detectors have full azimuthal coverage.

    Trajectory reconstruction is performed using Drift Chambers (DC) in the forward direction, achieving a momentum resolution of less than $1\%$.
    In the central detector, trajectory reconstruction is done using a vertex tracker, resulting in a momentum resolution of less than $3\%$.
    Particle identification relies on Cherenkov counters, time-of-flight scintillators, and electromagnetic calorimeters \cite{burkert2020}.
    Fast triggering and high data-acquisition rates enable operation at a luminosity of $10^{35} \text{ cm}^{-2}\text{ s}^{-1}$ \cite{burkert2020}.

    A diagram of CLAS12 showing the position of each detector component is provided in Figure \ref{fig::11.200::clas12_diagram}.

    % !TEX root = ../main.tex
\subsubsection{Forward Detector}
\label{sssec::forwarddetector}
    % The Forward Detector (FD) is an essential component of the CLAS12 spectrometer designed to detect particles scattered at small polar angles in the forward direction.
    % It consists of several subdetectors that play crucial roles in particle identification, tracking, and timing measurements.
    %
    % Based on its polar coverage, the FD can be divided into two: the Forward Tagger (FT) and the FD proper.
    % The former detects particles with a polar angle between $2.5\degree$ and $4.5\degree$.
    % The latter detects those with a polar angle between $5\degree$ and $35\degree$.
    % A detailed description of each subdetector systems is provided in the following paragraphs.
    %
    % The Forwards Micromegas Tracker (FMT) is part of the FD, but is not included in this list.
    % The detectors is explored in detail in Section \ref{ssec::forwardsmicromegastracker}.

    % !TEX root = ../main.tex
\paragraph{High Threshold Cherenkov Counter (HTCC)}
\label{par::htcc}
    \begin{wrapfigure}{l}{0.50\textwidth}
        \centering\frame{
        \includegraphics[width=\linewidth]{211htcc.png}}
        \caption[HTCC]{Render of the High Threshold Cherenkov Counter.
        The container spans a diameter of about 4.5 m. The mirror is seen at the downstream end to the right.
        The PMTs are mounted in 12 sectors and in groups of 4 at the outer perimeter of the container.
        Light collection uses additional Winston cones and 5-in PMTs with quartz windows.
        Source: \hyperlink{jlab.org/physics/hall-b/clas12}{CLAS12 wiki}.}
        \label{fig::htcc}
    \end{wrapfigure}

    The HTCC is specifically designed to separate electrons and positrons with momenta below 4.9 GeV from other charged particles.
    It achieves this through its capability for electron/positron identification, which provides high rejection of charged pions and low background noise.
    This is crucial for reliably identifying scattered electrons in an environment with a dense electromagnetic background.

    The HTCC is positioned downstream of the target and is fitted in between magnets, upstream of the forward tracking detectors.
    It ensures full azimuthal coverage, meaning it can detect particles emitted from any direction around its circumference.
    In terms of the polar angle, it spans from $5\degree$ to $35\degree$, covering a specific range of particle emission angles.
    Importantly, it has no blind areas in its complete solid angle coverage, meaning there are no regions where particles cannot be detected.

    Operating in dry CO2 gas at 1 atm pressure, the HTCC consists of a multi-focal mirror composed of 48 elliptical mirror facets.
    This mirror design enables the focusing of Cherenkov light produced by charged particles passing through the detector.
    The focused light is then detected by 48 Photomultiplier Tubes (PMTs), with each PMT featuring a quartz window of 125 mm in diameter.

    The PMTs are positioned within a magnetic field of up to $3.5\cdot 10^{-3}$ T, which is oriented along the axes of the phototubes.
    To minimize the impact of the magnetic field on the PMTs, they are surrounded along their lengths by a multi-layer magnetic shield.
    This shield includes active compensation coils, which further help in shielding the PMTs from the effects of the magnetic field.

    To minimize the effects of multiple scattering and its impact on the momentum analysis of charged tracks in the torus field, the HTCC mirror system is constructed using a backing structure made of low-density composite material.
    This choice of material helps to reduce the scattering of particles passing through the HTCC, thereby improving the accuracy of momentum measurements.

    Since the HTCC is located in front of the momentum analyzing torus magnet, it is important to minimize the presence of materials in the path of charged particles, except for the radiator gas.
    This is done to prevent interactions and disturbances that could affect the accuracy of momentum analysis.
    The HTCC is designed with this consideration in mind.

    In the HTCC, the density of solid material encountered by charged particles passing through its volume is approximately $135~\text{mg/cm}^2$.
    This low-density configuration ensures that the material contribution to multiple scattering is minimised, allowing for more precise momentum measurements of charged tracks.

    The HTCC also serves the purpose of generating a fast signal that can be used as a trigger for scattered electrons.
    This signal is utilised to identify and select scattered electrons for further analysis.

    In conjunction with the energy deposited in the electromagnetic calorimeters, the HTCC plays a role in the identification of electrons with specific energies.
    By combining the information from the HTCC and the electromagnetic calorimeters, the experiment can accurately identify electrons of interest based on their energy deposition patterns.

    A visual representation of the HTCC can be seen in Figure \ref{fig::htcc}, which provides a cut view of the detector and its components.

    Overall, the HTCC plays a crucial role in electron/positron identification by using a multi-focal mirror, PMTs, and a magnetic field setup.
    These components work together to ensure efficient detection and separation of electrons and positrons from other charged particles in a high-energy physics experiment environment.

    % !TEX root = ../main.tex
\paragraph{Drift Chambers (DC)}
    \begin{wrapfigure}{r}{0.50\textwidth}
        \centering\frame{
        \includegraphics[width=\linewidth]{212dc.png}}
        \caption[DC]{Drift Chambers render.
        Each of the DC regions are denoted as R1, R2, and R3 in the figure.}
        \label{fig::dc}
    \end{wrapfigure}

    The six coils of the torus magnet act as support for the forward tracking system, which consists of three independent drift chambers in each of the six sectors of the torus magnet.
    Each of the six DC sectors has a total of 36 layers with 112 sense wires each, arranged in three regions of twelve layers each.
    In each of the six torus sectors, the drift chambers are arranged identically.
    The arrangement of the DC around the torus coil can be seen in Figure \ref{fig::dc}

    As can be seen on the figure, the first region is located at the entrance to the torus magnetic field region.
    Then, the second is inside the magnet, where the magnetic field is close to its maximum.
    The third is located in a low magnetic field space, just downstream of the torus magnet.
    This arrangement provides independent and redundant tracking in each of the six torus sectors.

    Each of the three regions consists of six ``superlayers'', each of which contains two layers.
    One layer has wires strung at a stereo angle of $+6\degree$ while the second one of $-6\degree$, both with respect to the sector midplane.
    This stereo view enables excellent resolution in the polar angle ($\Delta\theta < 2 ~\text{mrad}$), and good resolution in the azimuthal scattering angle ($\Delta\phi < 2 ~\text{mrad}$).

    The DC can detect ionising particles with momenta above $200 ~\text{MeV}/\text{c}$, with a $\Delta p/p$ lesser than $0.5\%$.
    This offers a track momentum resolution of $3$ to $5\%$ \cite{mestayer2020}.

    % !TEX root = ../main.tex
\paragraph{Low Threshold Cherenkov Counter (LTCC)}
    The LTCC system is used for charged pion and kaon detection at momenta between $3.5$ and $9 ~\text{GeV}$.
    The LTCC system consists of boxes shaped like truncated pyramids.
    Four of the six sectors of CLAS12 are equipped with one LTCC box.
    Each LTCC box contains 108 lightweight mirrors with composite backing structures, 36 Winston light-collecting cones, 36 125-mm diameter PMTs, and 36 magnetic shields.
    The LTCC boxes are filled with heavy C4 F10 radiator gas.

    \begin{wrapfigure}{l}{0.50\textwidth}
        \centering\frame{
        \includegraphics[width=\linewidth]{213ltcc.png}}
        \caption[LTCC Mirror System]{Layout and components of the optical mirror system within each LTCC box from the design model.}
        \label{fig::ltcc}
    \end{wrapfigure}

    The LTCC was a detector used in CLAS, which as part of the 12 GeV upgrade was refurbished to provide higher efficiency for charged pion and kaon detection.
    This was done by increasing the volume of the radiator gas, refurbishing the elliptical and hyperbolic mirrors with new coatings, and improving the sensitivity of the PMTs to Cherenkov light.
    The sensitivity improvement was achieved by coating their entrance windows with wavelength shifting material that absorbs ultraviolet (UV) light at wavelength below $300 ~\text{nm}$ and re-emits two back-to-back photons at larger wavelength \cite{ungaro2020}.
    A drawing from the design model of the LTCC can be seen in Figure \ref{fig::ltcc}.

    % !TEX root = ../main.tex
\paragraph{Forward Time-of-Flight (FTOF)}
    \begin{wrapfigure}{r}{0.50\textwidth}
        \centering\frame{
        \includegraphics[width=\linewidth]{11experiment/img/214ftof.png}}
        \caption[FTOF]{Render of the Forward Carriage with the FTOF system showing the panel-1b counters on the inside (dark blue), and the panel-2 counters on the outside (bronze).
        The panel-1a counters are located immediately downstream of the panel-1b counters and are not visible in the render.
        Part of the PCAL is visible downstream of the FTOF panels.}
        \label{fig::ftof}
    \end{wrapfigure}

    The FTOF system is used to measure the TOF of charged particles emerging from the target during beam operation.
    It includes six sectors of plastic scintillators with double-sided PMT readout.
    Each sector consists of three arrays of counters separated in panels, with panel-1a having 23 counters, panel-1b 62 counters, and panel-2 5 counters.
    The system is required to get excellent timing resolution required for particle identification and good segmentation to get flexible triggering options.

    The detectors span a range in polar angle from $5\degree$ to $45\degree$, covering $50\%$ in the azimuth at $5\degree$ and $90\%$ at $45\degree$.
    The lengths of the counters range from $32.3 ~\text{cm}$ to $376.1 ~\text{cm}$ in panel 1a, from $17.3 ~\text{cm}$ to $407.9 ~\text{cm}$ in panel-1b, and from $371.3 ~\text{cm}$ to $426.2 ~\text{cm}$ in panel-2.
    The average timing resolution in panel-1a is $125 ~\text{ps}$, $85 ~\text{ps}$ in panel-1b, and $155 ~\text{ps}$ in panel-2 \cite{carman2020ftof}.
    A render of the detector can be seen in Figure \ref{fig::ftof}.

    % !TEX root = ../main.tex
\paragraph{Ring Imaging Cherenkov Detector (RICH)}
    For momenta greater than 3 GeV, the TOF resolution of FTOF is not sufficient to separate kaons from pions.
    For that puspose, an additional RICH detector was built and incorporated into one of the CLAS12 sectors to replace the corresponding LTCC sector.
    The RICH detector is designed to improve CLAS12 particle identification in the momentum range $3 - 8 ~\text{GeV}$.
    The detector incorporates aerogel radiators, visible light photon detectors, and a focusing mirror system that is used to reduce the detection area instrumented by photon detectors to $1 ~\text{m}^2$.

    Multi-anode PhotoMultiplier Tubes (MaPMTs) provide the required spatial resolution and match the aerogel Cherenkov light spectrum in the visible and near-UV region.
    For forward scattered particles up to $13\degree$ with momenta $3 - 8 ~\text{GeV}$, a proximity imaging method with thin ($2 ~\text{cm}$) aerogel and direct Cherenkov light detection is used.
    For larger incident particle angles between $13\degree$ and $25\degree$ and momenta of $3 - 6 ~\text{GeV}$, the Cherenkov light is produced by a thicker aerogel layer of $6 ~\text{cm}$, focused by a spherical mirror, and undergoes two further passes through the thin radiator material and a reflection from planar mirrors before detection \cite{contalbrigo2020}.

    % !TEX root = ../main.tex
\paragraph{Electromagnetic Calorimeters (ECAL)}
    The CLAS12 detector package uses the existing electromagnetic calorimeter (EC) of the CLAS detector, but adds to it a new pre-shower calorimeter (PCAL), which is installed upstream of the EC.
    Together, the PCAL and EC are referred to as the ECAL.
    The calorimeters in CLAS12 are used primarily for the identification and kinematical reconstruction of electrons, photons, and neutrons.

    The PCAL and EC are both sampling calorimeters consisting of six modules.
    Along the direction from the target, the EC consists of two parts, read out separately, called EC-inner (ECIN) and EC-outer (ECOU).
    They provide longitudinal sampling of electromagnetic showers, as well as of hadronic interactions to improve particle identification.
    Each module has a triangular shape with 54 (15/15/24, PCAL/ECIN/ECOU) layers of 1-cm-thick scintillators segmented into 4.5/10-cm (PCAL/EC) wide strips fitted between 2.2-mm-thick lead sheets.
    The total thickness corresponds to approximately 20.5 radiation lengths.
    Scintillator layers are grouped into three readout views with 5/5/8 PCAL/ECIN/ECOU layers per view, providing spatial resolutions of less than $2 ~\text{cm}$ for energy clusters.
    The light from each scintillator readout group is routed to the PMTs via flexible optical fibres \cite{asryan2020}.

    % !TEX root = ../main.tex
\paragraph{Forward Tagger (FT)}
    The FT extends the capabilities of CLAS12 to detect electrons and photons at the very forward polar angles from $2.5\degree$ to $4.5\degree$.
    The detection of forward-going scattered electrons allows for electroproduction experiments at very low photon virtuality $Q^2$, providing an energy-tagged, linearly polarized, high-intensity, quasi-real photon beam.
    This configuration enables execution of an extensive hadron spectroscopy program.

    \begin{wrapfigure}{l}{0.50\textwidth}
        \centering\frame{
        \includegraphics[width=\linewidth]{217ft.png}}
        \caption[FT]{The Forward Tagger system circled downstream of the CD in front of the torus magnet warm bore entrance.}
        \label{fig::ft}
    \end{wrapfigure}

    The FT consists of a calorimeter (FTCal), a micro-strip gas tracker (FTTrk), and a hodoscope (FTHodo).
    The FTCal with 332 lead-tungstate ($\text{PbWO}_4$) crystals is used to identify electrons, measure the electromagnetic shower energy, and provide a fast trigger signal.
    The FTTrk in front of it measures the charged particle scattering angles.
    The scintillator FTHodo aids in separating electrons and high-energy photons.

    During beam operations, a tungsten shielding pipe of conical shape is installed in front of the FT to absorb M\o ller electrons and low-energy photons produced by beam interactions with the target and downstream materials.
    This shield protects both the FT and the Forward Detectors from electromagnetic background.
    The cone angle is $2.5\degree$, such that it is compatible with the FT acceptance.
    In this configuration, known as ``FT-ON'', the FT can be used to detect both electrons and photons, extending the detection capabilities of CLAS12.

    Alternatively, when the FT is not needed for the physics program, the FT detectors are turned off and additional shielding elements are installed in front of the FT covering up to $4.5\degree$ to reduce the background in the DC R1 chambers.
    This configuration, known as ``FT-Off'', reduces the accidental background by one-third at the same beam conditions, which allows for higher luminosity data taking with CLAS12 \cite{acker2020ft}.
    A render of the CD with the FT circled can be seen in Figure \ref{fig::ft}.


    % !TEX root = ../main.tex
\subsubsection{Central Detector}
\label{sssec::centraldetector}
    In the CLAS12 Central Detector (CD), particles scattered from the target within the polar angle range of $35\degree$ to $125\degree$ are detected.
    The CD consists of various detectors that provide particle identification and tracking capabilities.
    Charged particles are detected in the Central Vertex Tracker (CVT) and the Central Time-of-Flight (CTOF) detector.
    Neutron detection is provided by the Central Neutron Detector (CND), which is located radially outside of the CVT and the CTOF.
    All detectors have full coverage in the azimuthal angle.

    % !TEX root = ../main.tex
\paragraph{Central Vertex Tracker (CVT)}
    The CVT system is an integral part of the CD.
    It is primarily used for measuring the momentum and determining the vertex position of charged particles scattered from the target.

    The CVT system is located inside the solenoid magnet, as depicted in Figure \ref{fig::11.221::cvt}.
    It consists of two distinct detectors: the SVT and the Barrel Micromegas Tracker (BMT).

    The SVT system is composed of three regions, each equipped with double-sided modules of silicon sensors.
    The regions have different numbers of modules: 10, 14, and 18, respectively.
    These silicon sensors are instrumented with a digital Application-Specific Integrated Circuit (ASIC) readout.
    The readout pitch, which refers to the distance between adjacent readout channels, is approximately 156 micrometers.
    Overall, the SVT system comprises 21,504 channels \cite{antonioli2020}.

    \begin{wrapfigure}{r}{0.50\textwidth}
        \centering\frame{
        \includegraphics[width=\linewidth]{221cvt.png}}
        \caption[CVT]{Render of the Central Vertex Tracker.
        From the inside, the figure shows the target cell and vacuum chamber, the three double layers of the SVT, followed by the six layers of the BMT.
        The beam enters from the left.
        The six Forwards Micromegas Tracker (FMT) layers are shown at the downstream end at the right.
        Source: \hyperlink{jlab.org/physics/hall-b/clas12}{CLAS12 wiki}.}
        \label{fig::11.221::cvt}
    \end{wrapfigure}

    The BMT is composed of three layers of strips aligned along the beamline and three layers of circular readout strips around the beamline, totalling 15,000 readout elements.
    It significantly enhances momentum resolution and tracking efficiency.
    Each layer is divided azimuthally into three segments, providing $120\degree$ azimuthal coverage for each segment.
    The system is designed to operate at the full luminosity of $10^{35} \text{cm}^{-2}\text{s}^{-1}$ \cite{acker2020mvt}.

    % !TEX root = ../main.tex
\paragraph{Central Time-of-Flight (CTOF)}
    The CTOF system is utilised for the purpose of charged particle identification by measuring their time-of-flight (TOF) in the momentum range of approximately $0.3$ to $1.25 ~\text{GeV}$.
    It consists of 48 plastic scintillators with double-sided PMT readout.
    The PMTs are connected to the scintillators via 1.0-meter-long upstream and 1.6-meter-long downstream focusing light guides.
    These counters are arranged in a hermetic barrel configuration surrounding the target and the CVT, and they are aligned with the beam axis inside the 5 T solenoid magnet.

    To ensure accurate measurements, the PMTs are positioned within a region of 0.1 T fringe field of the solenoid magnet and are enclosed within a triple-layer dynamic magnetic shield.
    This shield minimises the internal magnetic field near the PMT photocathode, achieving a field strength of less than 0.2 G.
    The CTOF system is designed to provide a time resolution of 80 ps, enabling precise charged particle identification in the CD \cite{carman2020ctof}.

    For a visual representation of the CTOF system, you can refer to Figure \ref{fig::11.222::ctof}.

    % !TEX root = ../main.tex
\paragraph{Central Neutron Detector (CND)}
    The Central Neutron Detector (CND) is a component of the CLAS12 Central Detector (CD) positioned radially outward of the CTOF system.
    Its primary function is to detect neutrons in the momentum range of $0.2$ to $1.0 \text{GeV}$ by measuring their TOF from the target and the energy deposition in scintillator layers.

    The CND is composed of three layers of scintillator paddles, with each layer containing 48 paddles.
    The paddles are coupled in pairs at the downstream end using semi-circular light guides.
    The signal generated in the scintillator paddles is read out at the upstream end by PMTs that are positioned outside the high magnetic field region of the solenoid magnet.

    To transmit the scintillation light, the scintillators are connected to 1-meter-long bent light guides, which ensure efficient light propagation to the PMTs for signal detection and readout.

    The combination of TOF measurements and energy deposition in the scintillator layers enables the CND to identify and detect neutrons within the specified momentum range in the CLAS12 CD \cite{chatagnon2020}.

    % !TEX root = ../main.tex
\paragraph{Back Angle Neutron Detector (BAND)}
    \begin{wrapfigure}{r}{0.49\textwidth}
        \frame{\includegraphics[width=\linewidth]{222ctof.png}}
        \caption[Central Time-of-Flight]
        {Render of the Central Time-of-Flight (CTOF).
        The render shows CTOF's 48 scintillator bars outfitted with light guides, PMTs, and magnetic shields at both ends of each counter.}
        \floatfoot{Source: \href{https://jlab.org/physics/hall-b/clas12}{CLAS12 wiki}.}
        \label{fig::11.222::ctof}
    \end{wrapfigure}

    The CLAS12 spectrometer includes the BAND as a dedicated detector for neutron detection at backward angles.
    Positioned 3 metres upstream of the target, the BAND is designed to detect backward-scattered neutrons with momenta ranging from 0.25 to 0.7 GeV.

    The BAND detector consists of 18 horizontal rows and five layers of scintillator bars.
    Each scintillator bar is equipped with PMT readout on both ends to measure the TOF of neutrons originating from the target.
    Additionally, there is an extra 1 cm scintillation layer specifically designed to veto charged particles, ensuring that only neutrons are detected.

    Covering a polar angle range from $155\degree$ to $175\degree$, the BAND detector achieves a design neutron detection efficiency of $35\%$.
    It also provides a momentum resolution of approximately $1.5\%$, allowing for precise measurements of the momentum of the detected neutrons \cite{segarra2020}.

    By utilising the BAND detector, the CLAS12 spectrometer is capable of detecting and characterising backward-scattered neutrons, providing valuable information for various physics studies and experiments conducted at CLAS12.


    % !TEX root = ../main.tex
\subsubsection{Offline Reconstruction}
\label{sssec::offline_reconstruction}
% --+ CLARA +-------------------------------------------------------------------
    The CLAS12 reconstruction and analysis process is facilitated by a data-stream processing framework called CLARA.
    CLARA adopts a service-oriented architecture, allowing the construction of software applications using micro-services that are connected via data-stream pipes \cite{gyurgyan2016}.

    In this framework, each service plays a specific role.
    It receives input data, processes it according to its functionality, and produces output data.
    The input and output data are organised in tabular structures known as ``banks'', which are configured by the service developer to match the specific requirements of the service.

    The services within CLARA form a data-flow path, where the output of one service becomes the input for the next service in the sequence.
    This design enables a flexible and versatile data processing application, as each service can be individually improved or replaced without necessitating structural changes to the framework.

    To ensure consistency and modularity, the CLAS12 services are extensions of an abstract reconstruction engine.
    This engine provides common components such as initialisation and event processing methods, reducing the development complexity of individual micro-services and enforcing a uniform structure throughout the framework.

    By leveraging the CLARA framework, the CLAS12 experiment benefits from a modular and adaptable data processing pipeline, allowing for efficient reconstruction and analysis of the collected data.
    The service-oriented architecture and data-stream processing approach contribute to the flexibility, scalability, and maintainability of the CLAS12 software framework.

    \begin{figure}[b!]
        \centering\frame{
        \includegraphics[width=\textwidth]{230recon_chain.pdf}}
        \caption[CLAS12 Reconstruction Chain.]{Graphical representation of the CLAS12 interdependencies between services and banks.
        The I/O service reads events from the input file and distributes them to the reconstruction services chain for processing.
        Each service reads the relevant banks, applies the reconstruction algorithm, and provides output banks that are passed to the next service in the chain.
        The Event Builder (EB) service is executed as last in the chain; it collects the relevant banks from all CLAS12 subsystems services and produces event, particle, and detector response banks that are written to the output file.
        Source: Own elaboration.}
        \label{fig::recon_chain}
    \end{figure}

% --+ CLAS12 reconstruction +---------------------------------------------------
    The CLAS12 data reconstruction process involves data reader services that access decoded detector data stored in banks.
    Each entry in the bank represents a decoded detector hit and contains information such as sector, layer, component, order, and digitised data like signal charge, amplitude, time, or pedestal.

    During the decoding stage, similar bank structures are created for various quantities required for event reconstruction, including hits, clusters, and tracks.
    Reconstruction algorithms specific to each CLAS12 subsystem fill these banks.
    The data persistence service appends and writes these banks to a file for later analysis.

    The reconstruction algorithms are implemented as services that operate on input banks and produce output banks, which are then passed to subsequent algorithms in the reconstruction chain.
    The order in which the services are chained reflects the overall sequence of CLAS12 event reconstruction and the dependencies between subsystems.

    The first step is the reconstruction of charged particle tracks in the Central and Forward Detector tracking systems, based on the recorded hit positions in the respective detectors.
    This process is known as ``hit-based'' tracking.

    Simultaneously, hits recorded in other detectors are processed to reconstruct the energy and time of the associated particle interactions.
    The Event Builder (EB) service matches these reconstructed quantities with the tracks based on position and time information.
    Hits that are not matched to any track are retained as neutral particle candidates.
    The EB also determines the event ``start time'' and identifies the reconstructed particles.

    Once the event start time is determined, a second iteration of forward tracking, known as ``time-based'' tracking, can be performed.
    This iteration incorporates the drift times in the Drift Chambers, providing improved tracking precision \cite{ziegler2020}.

    An overview of the composition of reconstruction application services, depicting the dependencies between the services, can be found in Figure \ref{fig::recon_chain}.

    % !TEX root = ../main.tex
\paragraph{Tracking}
    Charged particle tracking is the key element of the CLAS12 event reconstruction.
    It is separated into the reconstruction of tracks in the central tracker system and the forward tracking system.

    In the forward region, the torus magnet bends charged particles inward toward the beamline or outward of it depending on their charge.
    At full nominal current, the $\int Bdl$ varies from $2 \text{Tm}$ at $5\degree$ to $0.5 \text{Tm}$ at $40\degree$.
    The forward tracking system in charge of tracking in this region is comprised of the Forward MicroMegas Tracker (FMT) and the Drift Chambers (DC).

    In the central region, the $5 \text{T}$ solenoidal magnetic field bends charged tracks into helices.
    In it, the central tracking system is comprised of the Silicon Vertex Tracker (SVT) and the Barrel MicroMegas Tracker (BMT), which together form the Central Vertex Tracker (CVT).

% --+ Reconstruction +----------------------------------------------------------
    \begin{wrapfigure}{l}{0.50\textwidth}
        \centering\frame{
        \includegraphics[width=\linewidth]{11experiment/img/23cvt_pres.png}}
        \caption[CVT momentum resolution vs. momentum.]{Momentum resolution vs. momentum of simulated protons in the CVT without background.}
        \label{fig::cvt_pres}
    \end{wrapfigure}

    For both systems, track reconstruction comprises algorithms for pattern recognition and track fitting. Hit objects, corresponding to the passage of a particle through a particular detector component, require the transformation of an electronic signal into a location of the track’s
    position in the detector subsystem geometry.
    A hit is defined as a detector element represented by a geometric object, for example, a line representing a strip in the central tracker.
    These objects then form the input to the pattern recognition algorithms.

    \begin{figure}[t]
        \centering\frame{
        \includegraphics[width=\textwidth]{11experiment/img/23ced_event.png}}
        \caption[Particle going through DC.]{Views from CLAS12 Event Display (ced) of charged particle tracks in the DC showing cut-views to highlight different pairs of sectors of the CLAS12 Forward Detector.
        The coloured detector elements are the registered hits and the orange lines are the result of track reconstruction using the hits in the DC.
        The coloured areas about the detectors represent the regions of magnetic field from the torus and the solenoid.
        In these views the beam is incident from the left and the target is located in the middle of the solenoid (at the left edge of the image).}
        \label{fig::ced_event}
    \end{figure}

    Pattern recognition involves the identification of clusters of hits and the determination of the spatial coordinates and corresponding uncertainties for the hits and clusters.
    At the pattern recognition stage, hits that are consistent with belonging to a trajectory (such as a particle track) are identified.
    This set of hits is then fit to the expected trajectory with their uncertainties, incorporating the knowledge of the detector material and the detailed magnetic field map.
    An illustration of a particle going through the DC can be seen in Figure \ref{fig::ced_event}.

% --+ Performance +-------------------------------------------------------------
    The momentum resolutions in the central and forward trackers as a function of momentum are shown in Figures \ref{fig::cvt_pres} and \ref{fig::dc_pres} respectively.
    The distributions are fit with a function of the form $\sqrt{a + bx^2 + c/(1 + d/x^2)}$.
    In both distributions, the worsening of the resolution at low momentum is due to multiple scattering effects.
    The resolution also worsens as a function of momentum after a minimum is reached due to poorer track curvature resolution.

    \begin{wrapfigure}{r}{0.50\textwidth}
        \centering\frame{
        \includegraphics[width=\linewidth]{11experiment/img/23dc_pres.png}}
        \caption[DC momentum resolution vs momentum.]{Momentum resolution vs. momentum in the DC evaluated using pions simulated at $\theta = 15\degree \pm 5\degree$ and at $\phi = 0 \pm 5\degree$ without background.}
        \label{fig::dc_pres}
    \end{wrapfigure}

    For central tracking, an average CVT reconstruction efficiency of $87.3\%$ is obtained from a simulated proton sample with momenta in the range from $0.5$ to $2.5 \text{GeV}$.
    A slight drop of efficiency is observed for tracks with momenta less than $600 \text{MeV}$.
    The higher curvature of small $\text{p}_\perp$ tracks results in an increase in inefficiency due to acceptance effects.
    The dominant source of inefficiency is the gaps between the sensitive volumes for the BMT and the SVT.

    For forward tracking, the momentum resolution in the DC is evaluated using tracks simulated at $\theta = 15\degree \pm 5\degree$ and at $\phi = 0 \pm 5\degree$.
    This is to ensure that most tracks are within the sensitive volume.
    Furthermore, the DC momentum resolution is correlated with the polar angle since the track curvature is determined from the magnetic field intensity, which is higher at lower angles in the torus field.
    These resolutions are obtained from a Monte Carlo sample that does not include out-of-time backgrounds or misalignments of the tracking volumes \cite{ziegler2020}.

    % !TEX root = ../main.tex
\paragraph{Particle Identification}
\label{par::particle_identification}
% --+ PDG PID +-----------------------------------------------------------------
    The Particle Identification (PID) numbering scheme described here was initially introduced by the Particle Data Group (PDG) in 1988 \cite{yost1988}.
    Its purpose is to facilitate communication and data exchange between different generators, simulators, and analysis packages employed in particle physics.
    The system underwent subsequent revisions and adaptations in 1998 to allow for the systematic inclusion of undiscovered and hypothetical particles \cite{particle1998}.
    The PID convention utilised in this thesis is based on the most up-to-date version available at the time of writing, as referenced from the 2020 Review of Particle Physics by the PDG \cite{particle2020review}.

% --+ The Event Builder +-------------------------------------------------------
    The Event Builder (EB) is a crucial component within the reconstruction chain, serving multiple functions:
    \begin{itemize}
        \item
            It gathers information from upstream services.
        \item
            It correlates information from sub-detectors to form particles.
        \item
            It implements a general particle identification scheme.
        \item
            It organises the resulting information into a standardised and persistent data bank structure.
    \end{itemize}

    The EB service is executed twice using identical algorithms, first employing hit-based tracks and subsequently using time-based tracks.
    As mentioned previously, the results obtained from the hit-based EB are utilised to initialise time-based tracking.

% --+ Forming particles +-------------------------------------------------------
    In the definition of a reconstructed charged particle within CLAS12, the EB assumes that each reconstructed track in both the FD and the CD will be assigned an identification.
    The corresponding responses from the calorimeter, scintillator, and Cherenkov detectors are then associated with that particle based on geometric coincidences between the detector responses and the track.
    Matching criteria are established, which correspond to the resolution of each specific detector.
    The geometric matching process relies on the Distance of Closest Approach (DOCA) between the track and the detector response.

    A similar procedure is employed for the creation of neutral particles, with the distinction that the seeding is presently performed using unassociated responses from the ECAL for the FD and the CND (or the BAND) for the CD, instead of using tracks.

% --+ Event start time +--------------------------------------------------------
    A start time is assigned to the entire event and serves as the precise reference time for all time-based particle identification procedures.
    The determination of the start time relies on the optimal charged particle candidate in the FD with an associated timing response from the FTOF detector.

    The EB assigns the start time based on the highest energy electron detected in the ECAL.
    If no electron is found in the ECAL, the EB then searches for a positron in the ECAL.
    In the absence of any lepton candidates, the next track in the priority list is a forward-going positive track, which is assumed to be a positive pion ($\pi^+$).
    Finally, if no forward-going positive track is identified, the EB searches for a forward-going negative track, assumed to be a negative pion ($\pi^-$).
    When searching for $\pi^+$ or $\pi^-$ tracks, only the candidate with the highest momentum within each group is considered.

    A parallel event start time is determined from the FT system to facilitate physics analyses and triggers specifically for events where the primary scattered electron is at very forward angles within the FT.

    In such cases, all combinations of charged particles in both the FT and the FD are taken into account.
    The particle in the FT is assumed to be an electron, while all possible hadron mass hypotheses are considered for the FD tracks.
    The combination that exhibits the best time coincidence is selected, and the timing of the resulting FT electron is used to assign the start time.

    A correction to the start time is subsequently applied using the Radio Frequency (RF) signal from the accelerator, in conjunction with the reconstructed event vertex position.
    This correction effectively aligns the event start time with the most accurate measurement of the beam-bunch arrival time at the target.

    The uncorrected measured vertex time of a particle, denoted as $t_v$, can be expressed as follows
    \begin{equation*}
        t_v = t - \frac{P_L}{\beta c},
    \end{equation*}

    Here, $t$ represents the measured time response (e.g., in a scintillator), $P_L$ is the path length between the primary interaction vertex and the corresponding response, and $\beta c$ denotes the speed of the particle.

    Next, we calculate the time difference $\Delta t_{RF}$ between $t_v$ and the nearest beam bunch using the following formula
    \begin{equation*}
        \Delta t_{RF} = t_v + \frac{z_0 - z_v}{c} - t_{RF} - \frac{N}{2f_{RF}},
    \end{equation*}

    In this equation, $z_v$ represents the $z$-coordinate of the event vertex position, $z_0$ is the reference position calibration at the center of the target, and $c$ denotes the speed of light in vacuum.
    $t_{RF}$ and $f_{RF}$ correspond to the measured and calibrated RF time and frequency of the accelerator.
    These values can either be $2.004 \text{ ns}$ and $249.5 \text{ MHz}$, or $4.008 \text{ ns}$ and $499 \text{ MHz}$, respectively.
    During the reconstruction process, these values are obtained from the Run Conditions Database (RCDB).

    Subsequently, the time can be further corrected to the nearest beam bunch using the equation
    \begin{equation*}
        \Delta t^\prime_{RF} = \text{mod}\left(\Delta t_{RF}, \frac{1}{f_{RF}}\right) - \frac{1}{2f_{RF}},
    \end{equation*}

    This correction allows for RF-correction to $t_v$. Thus, we obtain a final RF- and vertex-corrected start time for the event, defined as
    \begin{equation*}
        t' = t_v - \Delta t^\prime_{RF}.
    \end{equation*}

% --+ Charged particle identification +-----------------------------------------
    The subsequent step involves a basic particle identification scheme designed to be flexible enough to accommodate various physics analyses while retaining essential information for future refinement of the criteria.

    \begin{wrapfigure}{r}{0.49\textwidth}
        \centering\frame{
        \includegraphics[width=\linewidth]{232pos_pid.png}}
        \caption[Particle $\beta$ vs. momentum for positively charged tracks.]{Particle $\beta$ vs. momentum from simulation data for positively charged tracks with their start time from an electron in the FD (top plot) or in the FT (bottom plot).
        Source: \cite{ziegler2020}.}
        \label{fig::pos_pid}
    \end{wrapfigure}

    For charged particles, the identification process begins by utilising calorimetry and Cherenkov information to positively identify $e^-/e^+$ candidates in the FD.
    If the measured energy deposition aligns with the expected sampling fraction of the ECAL and the photoelectron response from the HTCC aligns with $\beta \sim 1$, the particle is assigned as an $e^-$ or $e^+$ based on the sign of curvature determined from forward tracking with the DC in the presence of the torus magnetic field.

    The remaining charged particles are then assumed to be hadrons and are assigned an identity solely based on timing information.
    The $p, K, \pi$ candidate with the smallest time residual is selected.
    This time residual is calculated as the difference between the measured flight time of the particle and the flight time computed for a specific mass hypothesis.

    Figure \ref{fig::pos_pid} presents the $\beta$ vs. momentum distributions for forward-going positively charged hadrons reconstructed using data from the FTOF and DC subsystems.
    The electron is reconstructed either in the FD (top) or in the FT (bottom).
    The computed curves for different mass hypotheses are superimposed on the distributions.

    \begin{wrapfigure}{r}{0.50\textwidth}
        \centering\frame{
        \includegraphics[width=\linewidth]{232n_gamma.png}}
        \caption[$\beta$ distribution of neutrals.]{$\beta$ distribution for neutral particles as measured by the ECAL from simulation data, showing a sharp peak at $\beta = 1$ from photons and a broader, slower distribution from neutrons.
        Source: \cite{ziegler2020}.
        }
        \label{fig::n_gamma}
    \end{wrapfigure}

% --+ Neutral particle identification +-----------------------------------------
    To identify neutral particles, the analysis assumes the presence of only neutrons and photons, distinguished solely by timing and topological information.
    In the FD, the identification is based on the ECAL, while in the CD, it relies on the CND.
    The reconstructed cluster positions of these detectors are used to calculate the particle's travel path from the event vertex, assuming a straight-line trajectory.

    If the measured $\beta$ value is close to $1$, the particle is identified as a photon; otherwise, it is identified as a neutron.
    For photons in the FD, the momentum is determined from the deposited energy and the ECAL sampling fraction \cite{asryan2020}.
    For neutrons, the momentum is assigned based on the measured $\beta$, assuming the mass of a neutron.

    Figure \ref{fig::n_gamma} illustrates an example of the reconstructed $\beta$ values for neutral particles in the FD, demonstrating the separation between photons and neutrons.

% --+ Particle identification performance +-------------------------------------
    A particle identification quality factor, represented as a signed-$\chi^2$ or pull, is assigned based on the contributions from individual detector subsystem responses and their resolutions.
    For $e^-/e^+$ identification, the resolution-normalised distance from the expected ECAL sampling fraction is utilised, while for charged hadrons, the resolution-normalised time difference is employed.
    The resulting information is organised into standardised output bank structures for physics analysis.
    This includes the particle four-vectors, associated detector responses, and global event information such as beam RF and helicity details.

    The accuracy of the currently implemented particle identification algorithm can be estimated by comparing the assigned particle identification with the true identification in Monte Carlo simulations.
    Table \ref{tab::rpid} presents the particle identification matrix for the FD (left) and CD (right).
    The values are derived from simulations involving electron-hadron or electron-photon pairs with hadron and photon momenta ranging from $1$ to $2.5 \text{ GeV}$ and electron momenta ranging from $1$ to $9 \text{ GeV}$.
    The diagonal elements represent cases where the particle is correctly identified, while the off-diagonal elements represent cases of misidentification \cite{ziegler2020}.

    \begin{table}
        \caption{Particle identification matrix for the FD (left matrix) and CD (right matrix).
        The FD matrix is based on simulated hadrons and photons with momentum between $1$ and $2.5 \text{ GeV}$, and electrons up to $9 \text{ GeV}$.
        The CD matrix is based on simulated hadrons with momentum between $0.3$ and $1.1 \text{ GeV}$.
        The diagonal elements are correctly identified, while the off-diagonal elements are misidentified.
        Detector inefficiencies are included.}

        \begin{tabularx}{\textwidth}{XXXXXXX|XXXXX}
            \hline
                     & \multicolumn{6}{l}{\textit{FD Truth}} & \multicolumn{5}{l}{\textit{CD Truth}}  \\
            \cline{2-12}
                     & $e$      & $\pi$ & $K$  & $p$  & $n$  & $\gamma$ &       & $\pi$    & $K$  & $p$  & $n$  \\
            \hline
            $e$      & 0.98     &       &      &      &      &          &       &          &      &      &      \\
            $\pi$    &          & 0.93  & 0.10 & 0.00 &      &          & $\pi$ & 0.84     & 0.14 & 0.00 &      \\
            $K$      &          & 0.03  & 0.80 & 0.00 &      &          & $K$   & 0.11     & 0.80 & 0.01 &      \\
            $p$      &          & 0.03  & 0.02 & 0.98 &      &          & $p$   & 0.03     & 0.04 & 0.95 &      \\
            $n$      &          &       &      &      & 0.66 & 0.01     & $n$   &          &      &      & 0.11 \\
            $\gamma$ &          &       &      &      & 0.14 & 0.95     &       &          &      &      &      \\
            \hline
        \end{tabularx}
        \label{tab::rpid}
    \end{table}


