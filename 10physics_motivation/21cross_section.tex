% !TEX root = ../main.tex
\subsubsection{SIDIS Cross Section}
\label{sssec::sidis_cross_section}
    In SIDIS, it's common to assume that the timescale for the virtual photon absorption is much shorter than that of the quark to fragment into a hadron.
    Since it involves long-distance processes, this fragmentation process corresponds to very low $Q^2$, and is not calculable in perturbative QCD (pQCD).
    The fragmentation process in SIDIS is parametrised by fragmentation functions $D_f^h(Q^2, z)$.
    This measures the probability that an $f$-flavoured quark fragments into an $h$-type hadron with a fraction $z$ of the virtual photon energy ($Eh=z\nu$).
    In the quark-parton model, the cross section for $eN \rightarrow ehX$ is assumed to be the differential cross section from \eqref{eq::parton_model_cross_section} times the fragmentation probability just described.

    \begin{equation}
        \label{eq::fragmentation_probability}
        \frac{d^3\sigma(eN \rightarrow ehX)}{dxdQ^2dz} =
                \frac{d^2\sigma(eN \rightarrow eX)}{dxdQ^2} \cdot
                \frac{\sum_fe^2_f q_f(x,Q^2) D^h_f(Q^2,z)}{\sum_f e^2_f q_f(x,Q^2)}.
    \end{equation}

    We assume that the quasi-free scattering process and the fragmentary process are independent in the cross section.
    From \eqref{eq::fragmentation_probability}, the hadron multiplicity per DIS event is
    \begin{equation*}
        M_h(Q^2,z)
                \equiv \frac{1}{\sigma} \frac{d^3\sigma(eN \rightarrow ehX)}{dQ^2dz}
                = \frac{\int dx \sum_f e^2_f q_f(x,Q^2) D^h_f(Q^2,z)}{\int dx \sum_f e^2_f q_f(x,Q^2)},
    \end{equation*}
    where $\sigma$ is the differential inclusive DIS cross section $\frac{d^2\sigma(eN \rightarrow eX)}{dxdQ^2}$.
