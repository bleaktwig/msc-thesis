% !TEX root = ../main.tex
\subsection{Hadronisation in the Nuclear Medium}
\label{ssec::hadronisation_in_the_nuclear_medium}
    Hadronisation is the process in which quarks and gluons form hadrons.
    In the nuclear medium, this process is influenced by quark energy loss through two mediums: gluon radiation and multiple quark-nucleon scattering.
    Additionally, hadron-nucleon interactions also affect the process if hadronisation occurs inside the nucleus.

    The primary experimental observable in hadronisation is the multiplicity of hadrons produced in a dense nucleus compared to a light one, such as deuterium.
    In the absence of attenuation from interactions with the medium, these multiplicities should be identical, resulting in a ratio of unity.
    This ratio is known as the attenuation ratio and is defined as

    \begin{equation*}
        R^h_{\text{att}}(z,\nu) = \frac
            {\left( \frac{1}{\sigma} \frac{d^2\sigma(eN \rightarrow ehX)}{dzd\nu} \right)A}
            {\left( \frac{1}{\sigma} \frac{d^2\sigma(eN \rightarrow ehX)}{dzd\nu} \right){\prescript{2}{}{H}}},
    \end{equation*}

    where the derivative with respect to $Q^2$ is substituted with one with respect to $\nu$.
    The $\nu$ and $z$ dependence of this ratio can be used to study the nature of the hadron formation mechanism.

    One type of observable that can be isolated is the characteristic times for the distinct stages of the hadronisation process.
    The existence of these stages is dictated by two fundamental properties in QCD: confinement and causality.
    Confinement implies that a coloured quark can only propagate for a limited distance.
    Causality dictates that the equilibrium colour field of a hadron cannot be formed instantaneously.

    % !TEX root = ../main.tex
\subsubsection{Virtual Photon Absorption}
\label{10.31::virtual_photon_absorption}
    The first stage in the hadronisation process is the absorption of a virtual photon by a quark.
    This process occurs on a relatively short timescale, much less than 1 fm/c, and is governed by the wavelength of the virtual photon.

    During this stage, the virtual photon interacts with a quark, transferring energy and momentum to the quark.
    The quark undergoes a transition to a higher energy state, which initiates the subsequent stages of hadronisation.

    % !TEX root = ../main.tex
\subsubsection{Production Time}
\label{10.32::production_time}
    Then, there must be a stage in which the coloured quark propagates as a quasi-free particle.
    During this second stage, there is gluon radiation with a differential spectrum given by pQCD as
    \begin{equation*}
        d\omega^{q \rightarrow qg} =
            \frac{\alpha_s(k_\perp^2)}{4\pi}
            \frac{8}{3}\left[ 1 + \left( 1 - \frac{k}{E} \right)^2 \right]
            \frac{dk}{k} \frac{dk_\perp^2}{k_\perp^2},
    \end{equation*}
    where $E$ is the quark energy, $k$ is the 4-momentum of the gluon, and $k_\perp$ is its transverse momentum.
    The time associated with this stage is commonly referred to as the ``production time'' \cite{kopeliovich2004}, and it represents the duration in which the quark is deconfined.
    The production time is a characteristic of the propagating quark and should be independent of the final hadron formed.

    To a very good approximation, in deep inelastic kinematics and at $x > 0.1$, the struck quark absorbs all the energy of the virtual photon.
    Thus, the initial energy of the struck quark should be $\nu$.
    This quantity is much larger than the quark's mass (assuming an up or down quark), and thus we'll ignore it in the treatment of this theory.

    Conservation of energy then tells us that the final energy of the produced hadron should be no greater than $\nu$.
    Additionally, the gluon radiation leads to a loss of energy, resulting in the hadron's energy being below $\nu$.
    We can estimate this energy loss from the string model \cite{artru1974}.
    The main parameter in the string model is the string tension $\kappa \approx 1 \text{ GeV}/\text{fm}$.
    The growth of the string creates an energy loss governed by this $\kappa$.
    Therefore, we can estimate the rate of vacuum energy loss as
    \begin{equation*}
        \frac{dE}{dx}\Big|_\text{vacuum} \approx 1 \text{ GeV}/\text{fm}.
    \end{equation*}

    Then, if
    \begin{equation}
        z_h = \frac{E_h}{\nu},
        \label{eq::10.32::zh}
    \end{equation}
    is the fraction of the struck quark's energy retained by the hadron, then the term $\nu(1 - z_h)$ represents the energy loss through gluon radiation.

    Thus, an estimate of the distance of gluon emission is
    \begin{equation*}
        l_p = \frac{\nu(1 - z_h)}{\kappa},
    \end{equation*}
    and therefore the production time $\tau_p$ is $l_p/c$ \cite{kopeliovich2004}.

    As an example, for a 10 GeV pion with $z_h = 0.6$, $l_p = 4$ fm -- the production time is of the order of only a few fm for Jefferson Lab (JLab) energies.

    % !TEX root = ../main.tex
\subsubsection{Formation Time}
\label{10.33::formation_time}
    In the final stage of the hadronisation process, the struck quark finds partner quarks to neutralise its colour, thus ending the gluon radiation.
    This stage is characterised by the evolution of the pre-hadron to an ordinary hadron and is commonly named the ``formation time''.
    This is the time required to form the colour field of a hadron, and it should depend on the specific hadron being formed.

    The formation time is not directly related to confinement, but rather it is a measure of the time required to form the non-perturbative colour field of the hadron, starting from a small colour-singlet object.
    This field generation time has a well-known analogue in Quantum Electrodynamics (QED).

    We can build a simple estimate for the formation time.
    To form a hadron of radius $R$ from a point-like, single-colour singlet, the speed at which the field can arise (in its rest frame) is bound by the speed of light

    \begin{equation*}
        \tau^\text{rest}_\text{formation} > \frac{R}{c},
    \end{equation*}

    which, when Lorentz-boosted into the lab frame, becomes

    \begin{equation*}
        \tau^\text{lab}_\text{formation} > \frac{E}{m^*} \frac{R}{c},
    \end{equation*}

    where $m^*$ is the mass of the propagating colour-singlet object.
    In principle, $m^*$ ranges from the mass of two bare quarks to the fully formed hadron mass $m_h$.
    A lower limit for the formation time is given by setting $m^* = m_h$.

    While this estimate is a classical calculation, a quantum mechanical analysis taking into account the gluon wavelength arrives at the same result, as is explained in \cite{dokshitzer1991}.

    % For a concrete example, for a 10 GeV pion with a mass of 0.139 GeV and a radius of 0.7 fm, the formation time should be at least 50.4 fm.
    % Thus, in total, one can expect a pion production time of 3.3 fm and a formation time of at least 50.4 fm in the lab frame at JLab energies.

