% !TEX root = ../main.tex
\subsection{Hadronisation in the Nuclear Medium}
% --+ Introduction +------------------------------------------------------------
    Hadronisation is the process in which quarks and gluons form hadrons.
    When in the nuclear medium, this process is influenced by quark energy loss through two mediums: Gluon radiation and multiple quark-nucleon scattering.
    Moreover, hadron-nucleon interactions also affect the process if the hadronisation happens inside the nucleus.

    In the process, the primary experimental observable is the multiplicity of hadrons produced on a dense nucleus as compared to a light one -- such as deuterium.
    In the absence of attenuation from interactions with the medium, these two quantities should be identical, such that the ratio between multiplicities would be unity.
    The ratio of hadron multiplicities -- known as attenuation ratio -- is defined as
    \begin{equation*}
        R^h_{\text{att}}(z,\nu) = \frac
                {\left( \frac{1}{\sigma} \frac{d^2\sigma(eN \rightarrow ehX)}{dzd\nu} \right)_A}
                {\left( \frac{1}{\sigma} \frac{d^2\sigma(eN \rightarrow ehX)}{dzd\nu} \right)_{\prescript{2}{}{H}}},
    \end{equation*}
    where the derivative with respect to $Q^2$ is substituted with one with respect to $\nu$.
    The $\nu$ and $z$ dependence of this ratio can be used to study the nature of the hadron formation mechanism.

    One type of observable that can be isolated is the characteristic times for the distinct stages of the hadronisation process.
    The existence of these stages is dictated by two of the most fundamental properties in QCD.
    First, confinement: a coloured quark can only propagate for a limited distance.
    Second, causality: the equilibrium colour field of a hadron cannot be formed instantaneously.

    \input{10physics_motivation/virtual_photon_absorption}
    \input{10physics_motivation/production_time}
    \input{10physics_motivation/formation_time}
