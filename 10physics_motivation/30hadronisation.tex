% !TEX root = ../main.tex
\subsection{Hadronisation in the Nuclear Medium}
% --+ Introduction +------------------------------------------------------------
    Hadronisation is the process in which quarks and gluons form hadrons.
    When in the nuclear medium, this process is influenced by quark energy loss through two mediums: Gluon radiation and multiple quark-nucleon scattering.
    Moreover, hadron-nucleon interactions also affect the process if the hadronisation happens inside the nucleus.

    In the process, the primary experimental observable is the multiplicity of hadrons produced on a dense nucleus as compared to a light one -- such as deuterium.
    In the absence of attenuation from interactions with the medium, these two quantities should be identical, such that the ratio between multiplicities would be unity.
    The ratio of hadron multiplicities -- known as attenuation ratio -- is defined as
    \begin{equation*}
        R^h_{\text{att}}(z,\nu) = \frac
                {\left( \frac{1}{\sigma} \frac{d^2\sigma(eN \rightarrow ehX)}{dzd\nu} \right)_A}
                {\left( \frac{1}{\sigma} \frac{d^2\sigma(eN \rightarrow ehX)}{dzd\nu} \right)_{\prescript{2}{}{H}}},
    \end{equation*}
    where the derivative with respect to $Q^2$ is substituted with one with respect to $\nu$.
    The $\nu$ and $z$ dependence of this ratio can be used to study the nature of the hadron formation mechanism.

    One type of observable that can be isolated is the characteristic times for the distinct stages of the hadronisation process.
    The existence of these stages is dictated by two of the most fundamental properties in QCD.
    First, confinement: a coloured quark can only propagate for a limited distance.
    Second, causality: the equilibrium colour field of a hadron cannot be formed instantaneously.

% --+ Absorption of the virtual photon +----------------------------------------
    \subsubsection{Virtual Photon Absorption}
        First, there is the absorption of a virtual photon by a quark on a presumably brief time, $\ll 1 \text{ fm}/\text{c}$, governed by the virtual photon wavelength.

% --+ Production time +---------------------------------------------------------
    \subsubsection{Production Time}
        Then, there must be a stage in which the coloured quark propagates as a quasi-free particle.
        During this second stage there is a gluon radiation with a differential spectrum given by pQCD as
        \begin{equation*}
            d\omega^{q \rightarrow qg} =
                    \frac{\alpha_s(k_\perp^2)}{4\pi}
                    \frac{8}{3}\left[ 1 + \left( 1 - \frac{k}{E} \right)^2 \right]
                    \frac{dk}{k} \frac{dk_\perp^2}{k_\perp^2},
        \end{equation*}
        where $E$ is the quark energy, $k$ is the 4-momentum of the gluon, and $k_\perp$ its transverse momentum.
        The time associated to this stage is commonly referred to as the \textit{production time} \cite{kopeliovich2004}, and is the length of time in which the quark is deconfined.
        Production time is a characteristic of the propagating quark, and should be independent of the final hadron formed.

        To a very good approximation, in deep inelastic kinematics and at $x > 0.1$, the struck quark absorbs all the energy of the virtual photon.
        Thus, the initial energy of the struck quark should be $\nu$.
        This quantity is much larger than the quark's mass (assuming an up or down quark), and thus we'll ignore it in the treatment of this theory.

        Conservation of energy then tells us that the final energy of the hadron produced should be no greater than $\nu$.
        Additionally, the gluon radiation leads to a loss of energy, meaning that the hadron's energy should be below $\nu$.
        We can estimate this energy loss from the string model \cite{artru1974}.
        Its main parameter is the string tension $\kappa \approx 1 \text{ GeV}/\text{fm}$.
        The growth of the string creates an energy loss governed by this $\kappa$.
        Therefore, we can estimate the rate of vacuum energy loss as
        \begin{equation*}
            \frac{dE}{dx}\Big|_\text{vacuum} \approx 1 \text{ GeV}/\text{fm}.
        \end{equation*}

        If $z_h = E_h/\nu$ is the fraction of the struck quark's energy retained by the hadron, then $\nu(1 - z_h)$ is the energy loss through gluon radiation.
        Thus, an estimate of the distance of gluon emission is
        \begin{equation*}
            l_p = \frac{\nu(1 - z_h)}{\kappa},
        \end{equation*}
        and thus production time $\tau_p$ is $l_p/c$ \cite{kopeliovich2004}.

        As an example, for a $10$ GeV pion with $z_h = 0.6$, $l_p = 4$ fm -- the production time is of the order of only a few fm for JLab energies.

% --+ Formation time +----------------------------------------------------------
    \subsubsection{Formation Time}
        In the final stage of the hadronisation process, the struck quark finds partner quarks to neutralise its colour, thus ending the gluon radiation.
        This stage is characterised by the evolution of the pre-hadron to an ordinary hadron, and is commonly name
        The formation time is the time required to form the colour field of a hadron.
        This time should depend on the hadron being formed.

        Formation time is not directly related to confinement, but rather is a measure of the time required to form the non-perturbative colour field of the hadron, starting from a small colour-singlet object.
        This field generation time has a well known analogue in Quantum Electrodynamics (QED).

        We can build a simple estimate for the formation time.
        To form an hadron of radius $R$ from a point-like, single-colour singlet, the speed at which the field can arise (in its rest frame) is bound by the speed of light
        \begin{equation*}
            \tau^\text{rest}_\text{formation} > \frac{R}{c},
        \end{equation*}
        which, Lorentz-boosted into the lab frame, is
        \begin{equation*}
            \tau^\text{lab}_\text{formation} > \frac{E}{m^*} \frac{R}{c},
        \end{equation*}
        where $m^*$ is the mass of the propagating colour-singlet object.
        In principle, $m^*$ ranges from the mass of two bare quarks to the fully formed hadron mass $m_h$.
        Then, a lower limit for the formation time is given by setting $m^* = m_h$.

        While this estimate is a classical calculation, a quantum mechanical analysis taking into account gluon wavelength arrives at the same result \cite{dokshitzer1991}.

        For a concrete example, for a $10$ GeV pion -- mass of $0.139$ GeV and radius of $0.7$ fm --, the formation time should be at least $50.4$ fm.
        Thus, in total, one can expect a pion production time of $3.3$ fm and formation time of at least $50.4$ fm in the lab frame at JLab energies.
