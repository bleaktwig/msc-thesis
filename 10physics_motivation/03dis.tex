% --+ How to study particle physics and DIS +-----------------------------------
Deep Inelastic Scattering (DIS) is the process used to investigate the interior of hadrons using leptons.
The process is similar to Rutherford scattering and provided the first experimental evidence of quarks.
DIS can be employed to delve even deeper into the structure of matter by utilising increasingly higher energies, thanks to Werner Heisenberg's uncertainty principle.

High-energy probes lead to asymptotic freedom, which is the property where the interactions between particles, such as quarks, become increasingly weak at shorter distances.
This implies that inside hadrons, quarks mostly move as free, non-interacting particles.
This allows for reliable calculation of event cross-sections in particle physics.
