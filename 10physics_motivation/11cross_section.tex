% !TEX root = ../main.tex
\subsubsection{DIS Cross Section}
\label{sssec::dis_cross_section}
    To approximate the cross section of electron-nucleon scattering, we will work in the center of mass reference frame.
    In this frame, the electron and nucleon are moving towards each other with sufficient energy, allowing us to neglect the nucleon's mass.
    As a result, the nucleon possesses nearly lightlike momentum along the collision axis.
    Consequently, the constituent quarks of the nucleon also have nearly lightlike momenta that are nearly collinear to the nucleon's momentum.
    Hence, as a first-order approximation, we can express the quark's momentum as
    \begin{equation*}
        p = \xi P,
    \end{equation*}
    where $\xi$ represents the longitudinal fraction of the quark's momentum, and thus $0 < \xi < 1$.

    In the leading-order approximation, we can also disregard gluon emission and exchange during the collision.
    Therefore, the cross section of electron-nucleon scattering is equal to that of electron-quark scattering for a given $\xi$, multiplied by the probability that the nucleon contains a quark with a longitudinal momentum fraction of $\xi$, integrated over $\xi$.

    This calculation encounters the issue that the probability of a nucleon containing a quark with a specific momentum cannot be computed within perturbative QCD.
    It relies on the soft processes that determine the nucleon's structure as a composite system of quarks and gluons.
    Consequently, we must consider this probability as an unknown function that needs to be measured in experiments.

    Such probability functions are known as Parton Distribution Functions (PDFs).
    A PDF can be assigned to various types of quarks, antiquarks, and gluons, and is incorporated into the nucleon's wave function.
    For each parton $f$, its PDF is defined as
    \begin{equation*}
        P_f = f_f(\xi)d\xi.
    \end{equation*}
    Therefore, the cross section for the inelastic scattering of an electron off a nucleon, within the leading-order approximation, can be expressed as
    \begin{equation*}
        \sigma\left( e^-(k) p(P) \rightarrow e^-(k') X \right) =
                \int_0^1d\xi \sum_f f_f(\xi) \cdot
                \sigma\left( e^-(k) q_f(\xi P) \rightarrow e^-(k') + q_f(p') \right),
    \end{equation*}
    where $X$ denotes the final hadronic state.
    It is important to remember that this equation does not provide an exact QCD prediction but represents the first-term expansion of $\alpha_s$.
    This approximation is known as the parton model \cite{halzen1991}.
