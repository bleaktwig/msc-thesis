% !TEX root = ../main.tex
% --+ Approximating the cross section +-----------------------------------------
To approximate the cross section of electron-nucleon scattering, we will work in the center of mass reference frame.
In this frame, the electron and nucleon are moving towards each other with sufficient energy, allowing us to neglect the nucleon's mass.
As a result, the nucleon possesses nearly lightlike momentum along the collision axis.
Consequently, the constituent quarks of the nucleon also have nearly lightlike momenta that are nearly collinear to the nucleon's momentum.
Hence, as a first-order approximation, we can express the quark's momentum as
\begin{equation*}
    p = \xi P,
\end{equation*}
where $\xi$ represents the longitudinal fraction of the quark's momentum, and thus $0 < \xi < 1$.

In the leading-order approximation, we can also disregard gluon emission and exchange during the collision.
Therefore, the cross section of electron-nucleon scattering is equal to that of electron-quark scattering for a given $\xi$, multiplied by the probability that the nucleon contains a quark with a longitudinal momentum fraction of $\xi$, integrated over $\xi$.

This calculation has the problem that the probability that the nucleon contains a quark with a particular momentum can't be calculated in perturbative QCD.
It depends on the soft processes that define the structure of the nucleon as a bound state of quarks and gluons.
We must therefore consider this probability an unknown function to be measured in the experiment.

This kind of probability functions are called Parton Distribution Functions (PDF).
A PDF can be used for all kinds of quarks, antiquarks, and gluons, and are incorporated inside the nucleon's wave function.
For each parton $f$, its PDF is defined as
\begin{equation*}
    P_f = f_f(\xi)d\xi.
\end{equation*}
Therefore, the cross section of the inelastic scattering of the electron off the nucleon in leading order approximation is
\begin{equation*}
    \sigma\left( e^-(k) p(P) \rightarrow e^-(k') X \right) =
            \int_0^1d\xi \sum_f f_f(\xi) \cdot
            \sigma\left( e^-(k) q_f(\xi P) \rightarrow e^-(k') + q_f(p') \right)
\end{equation*}
where $X$ indicates the final hadronic state.
One should remember that this equation is not an exact QCD prediction, but is the first-term expansion of $\alpha_s$.
This approximation is the parton model \cite{halzen1991}.
