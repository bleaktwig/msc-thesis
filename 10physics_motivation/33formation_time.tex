% !TEX root = ../main.tex
\subsubsection{Formation Time}
\label{10.33::formation_time}
    In the final stage of the hadronisation process, the struck quark finds partner quarks to neutralise its colour, thus ending the gluon radiation.
    This stage is characterised by the evolution of the pre-hadron to an ordinary hadron and is commonly named the ``formation time''.
    This is the time required to form the colour field of a hadron, and it should depend on the specific hadron being formed.

    The formation time is not directly related to confinement, but rather it is a measure of the time required to form the non-perturbative colour field of the hadron, starting from a small colour-singlet object.
    This field generation time has a well-known analogue in Quantum Electrodynamics (QED).

    We can build a simple estimate for the formation time.
    To form a hadron of radius $R$ from a point-like, single-colour singlet, the speed at which the field can arise (in its rest frame) is bound by the speed of light

    \begin{equation*}
        \tau^\text{rest}_\text{formation} > \frac{R}{c},
    \end{equation*}

    which, when Lorentz-boosted into the lab frame, becomes

    \begin{equation*}
        \tau^\text{lab}_\text{formation} > \frac{E}{m^*} \frac{R}{c},
    \end{equation*}

    where $m^*$ is the mass of the propagating colour-singlet object.
    In principle, $m^*$ ranges from the mass of two bare quarks to the fully formed hadron mass $m_h$.
    A lower limit for the formation time is given by setting $m^* = m_h$.

    While this estimate is a classical calculation, a quantum mechanical analysis taking into account the gluon wavelength arrives at the same result, as is explained in \cite{dokshitzer1991}.

    % For a concrete example, for a 10 GeV pion with a mass of 0.139 GeV and a radius of 0.7 fm, the formation time should be at least 50.4 fm.
    % Thus, in total, one can expect a pion production time of 3.3 fm and a formation time of at least 50.4 fm in the lab frame at JLab energies.
