% !TEX root = ../main.tex
% --+ Classical particle physics +----------------------------------------------
Since ancient times, humanity has pondered the composition of matter.
Philosophers from both East and West have often contemplated this question, and the modern view of this composition is based on the concept of the atom.
The notion of the atom, or indivisible particle, was proposed by Democritus in the 6th century B.C.
Despite its uncanny similarity to the modern concept of the atom, the model remained an abstract idea for more than two millennia.

In the grand scheme of things, it is only recently that we have been able to probe into the structure of matter and observe the atom.
In 1897, J.J. Thomson discovered "corpuscles," or electrons, using a cathode ray tube.
Based on this discovery, he proposed an atomic model consisting of a positively charged paste with lighter electrons floating inside.
Then, in 1909, Ernest Rutherford put this model to the test in what became the first scattering experiment in history.
By bombarding \textalpha particles onto a thin gold foil, he proved that most of the atom's mass was concentrated in a small, positively charged nucleus at its center.
He named the constituents of this nucleus protons.

Following that, Niels Bohr proposed the atomic planetary model in 1914.
His theory precisely fit the experimental data for Hydrogen but did not apply to heavier atoms.
This issue was resolved with the discovery of the neutron in 1932 by James Chadwick.
This discovery made the masses of atoms consistent with available experimental data, marking the end of the era of classical particle physics.
