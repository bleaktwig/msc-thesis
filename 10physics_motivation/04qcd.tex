% !TEX root = ../main.tex
% --+ Quantum Chromodynamics +--------------------------------------------------
Another important feature is colour confinement, which is the property of colour-charged particles that prevents their isolation.
Colour charge is the Quantum Chromodynamics (QCD) equivalent of electric charge.
There are three colour charges and their corresponding anticharges.
Quarks possess a single colour charge, while gluons, the force-mediators of the strong force, have a bi-colour charge.
In a state of equilibrium, the strong force confines quarks to be in close proximity, forming quark-antiquark pairs or 3-quark triplets in such a way that the net colour charge is neutral.

QCD describes these properties and models the strong interaction between quarks.
Similar to the electromagnetic force, the strength of the interaction is determined by the strong coupling constant \textalpha.
However, unlike the electromagnetic force, this constant weakens as distances decrease.
