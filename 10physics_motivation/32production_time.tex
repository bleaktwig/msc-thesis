% !TEX root = ../main.tex
\subsubsection{Production Time}
\label{10.32::production_time}
    Then, there must be a stage in which the coloured quark propagates as a quasi-free particle.
    During this second stage, there is gluon radiation with a differential spectrum given by pQCD as

    \begin{equation*}
        d\omega^{q \rightarrow qg} =
            \frac{\alpha_s(k_\perp^2)}{4\pi}
            \frac{8}{3}\left[ 1 + \left( 1 - \frac{k}{E} \right)^2 \right]
            \frac{dk}{k} \frac{dk_\perp^2}{k_\perp^2},
    \end{equation*}

    where $E$ is the quark energy, $k$ is the 4-momentum of the gluon, and $k_\perp$ is its transverse momentum.
    The time associated with this stage is commonly referred to as the ``production time'' \cite{kopeliovich2004}, and it represents the duration in which the quark is deconfined.
    The production time is a characteristic of the propagating quark and should be independent of the final hadron formed.

    To a very good approximation, in deep inelastic kinematics and at $x > 0.1$, the struck quark absorbs all the energy of the virtual photon.
    Thus, the initial energy of the struck quark should be $\nu$.
    This quantity is much larger than the quark's mass (assuming an up or down quark), and thus we'll ignore it in the treatment of this theory.

    Conservation of energy then tells us that the final energy of the produced hadron should be no greater than $\nu$.
    Additionally, the gluon radiation leads to a loss of energy, resulting in the hadron's energy being below $\nu$.
    We can estimate this energy loss from the string model \cite{artru1974}.
    The main parameter in the string model is the string tension $\kappa \approx 1 \text{ GeV}/\text{fm}$.
    The growth of the string creates an energy loss governed by this $\kappa$.
    Therefore, we can estimate the rate of vacuum energy loss as

    \begin{equation*}
        \frac{dE}{dx}\Big|_\text{vacuum} \approx 1 \text{ GeV}/\text{fm}.
    \end{equation*}

    If $z_h = E_h/\nu$ is the fraction of the struck quark's energy retained by the hadron, then $\nu(1 - z_h)$ is the energy loss through gluon radiation.
    Thus, an estimate of the distance of gluon emission is

    \begin{equation*}
        l_p = \frac{\nu(1 - z_h)}{\kappa},
    \end{equation*}

    and therefore the production time $\tau_p$ is $l_p/c$ \cite{kopeliovich2004}.

    As an example, for a 10 GeV pion with $z_h = 0.6$, $l_p = 4$ fm -- the production time is of the order of only a few fm for JLab energies.
