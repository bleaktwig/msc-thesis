% !TEX root = ../main.tex
\subsubsection{\texttt{draw\_plots}}
    This executable is included so that the user doesn't have to re-write similar code regularly to obtain plots.
    Operating \texttt{draw\_plots} is fairly simple: after running the program, the user must answer a set of question to define the various attributes related to the plots.
    This includes cuts and corrections to apply, binning setup, and, naturally, the variables in the plots.
    Unless otherwise specified, the plots included in section \ref{sec::resultsandconclusions} were produced using this executable.

    The executable's manual entry is:
    \begin{lstlisting}
Usage: draw_plots [-hp:cn:o:a:w:] infile
 * -h          : show this message and exit.
 * -p pid      : skip particle selection and draw plots for pid.
 * -c          : apply all cuts (general, geometry, and DIS) instead of
                 asking which ones to apply while running.
 * -n nentries : number of entries to process.
 * -o outfile  : output file name. Default is plots_<run_no>.root.
 * -a accfile  : apply acceptance correction using acc_filename.
 * -A          : get acceptance correction plots without applying acceptance
                 correction. Requires -a to be set.
 * -w workdir  : location where output root files are to be stored. Default
                 is root_io.
 * infile      : input file produced by make_ntuples.\n

Draw plots from a ROOT file built from `make_ntuples`. File should be named `<text>run_no.root`. This tool is built for those who don't enjoy using root too much, and should be able to get most basic plots needed in SIDIS analysis.
    \end{lstlisting}
